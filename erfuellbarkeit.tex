\section{Erfüllbarkeit aussagenlogischer Formeln}
\begin{frame}{Entscheidungsproblem SAT}
	\begin{itemize}
		\item Ist eine gegebene aussagenlogische Formel erfüllbar / nicht erfüllbar?
		\item Gesucht sind \emph{Algorithmen} (Verfahren, Handlungsanweisungen), die für eine (beliebige) aussagenlogische Formel als Eingabe nach endlich vielen Schritten mit (korrekter) Aussage ja/nein terminieren.
		\item trivialer Algorithmus: Berechnung der Wahrheitstafel $\rightarrow$ nicht effizient
		\item Bemerkung: jede aussagenlogische Formel lässt sich effizient in eine erfüllbarkeitsäquivalente Formel in KNF umschreiben (unter Einführung zusätzlicher Variablen) $\rightarrow$ betrachte im weiteren Formeln in KNF
	\end{itemize}
\end{frame}

\begin{frame}{Klauselmengen}
	\begin{itemize}
		\item Mengennotation für Formeln in KNF\\
		ersetze Klausel $L_{i1} \lor L_{i2} \lor \ldots$ durch $C_i = \left\{L_{i1}, L_{i2}, \ldots\right\}$ und Formel $F$ in KNF durch Klauselmenge $S=\left\{C_1, C_2, \ldots\right\}$
		\item Belegung $I$ lässt sich ebenfalls als Menge darstellen:\\
		z.B. $I(x_1)=w, I(x_2)=f, \ldots$\\
		dann Schreibweise $I=\left\{x_1, \overline{x_2}, \ldots\right\}$
		\item Belegung $I$ ist Modell der Klausel $C_i$ gdw. $I \cap C_i \neq \varnothing$.\\
		Belegung $I$ ist Modell der Klauselmenge $S$ gdw. $I \cap C_i \neq \varnothing$ für alle $C_i \in S$.
	\end{itemize}
\end{frame}

\begin{frame}{Hornformeln}
	\begin{itemize}
		\item Definition: $F$ ist Hornformel gdw. $F$ ist in KNF und jede Klausel enthält höchstens ein positives Literal.
		\item Beispiel: $F_1=(A \lor \neg B) \land (\neg C \lor \neg A \lor D) \land (\neg A \lor \neg B) \land D \land \neg E$
		\item Es gilt: jede Hornformel lässt sich äquivalent in eine Konjunktion von Implikationen ("`Regeln"') umformen
		\item Beispiel: $F_1=(B \rightarrow A) \land (A \land C \rightarrow D) \land (A \land B \rightarrow f) \land (w \rightarrow D) \land (E \rightarrow f)$
		\item Eine Menge von Hornklauseln heißt auch Logikprogramm
		\begin{itemize}
			\item Tatsachenklausel: ein positives und kein negatives Literal
			\item Prozedurklausel (Regel): ein positives und mindestens ein negatives Literal
			\item Zielklausel (Frageklausel): negative Klausel (ohne positives Literal)
		\end{itemize}
	\end{itemize}
\end{frame}

\begin{frame}{Beweismethodik}
	\begin{itemize}
		\item Zeige Gültigkeit einer regelbasierten aussagenlogischen Formel durch Nichterfüllbarkeit von $\neg F$
		\item Beispiel: es gelten die Prämissen $B, B \rightarrow A$ und $(A \land B) \rightarrow C$; logisches Schließen liefert $A$ und daraus auch $C$; also ist\\
		$F=(B \land (B \rightarrow A) \land (A \land B \rightarrow C)) \rightarrow (A \land C)$\\
		eine Tautologie. Es gilt\\
		$\neg F = B \land (B \rightarrow A) \land (A \land B \rightarrow C) \land (\neg A \lor \neg C)$\\
		das ist Hornformel, mit Zielklausel $(\neg A \lor \neg C) \equiv ((A \land C) \rightarrow f)$
		\item Anstelle die Gültigkeit von F direkt zu beweisen, zeige die Nichterfüllbarkeit der Hornformel $\neg F$
	\end{itemize}
\end{frame}

\begin{frame}{Erfüllbarkeitstest (Markierungsalgorithmus)}
	Sei F Konjunktion von Implikationen
	\begin{enumerate}
		\item Für alle Teilformeln der Art $w \rightarrow A$: markiere in allen Teilformeln auftretende $A$
		\item while (es gibt Teilformeln der Art
		\begin{enumerate}
			\item[(i)] $A_1 \land A_2 \land \ldots \land A_n \rightarrow B$ bzw. $1 \rightarrow B$ oder
			\item[(ii)] $A_1 \land A_2 \land \ldots \land A_n \rightarrow f$\\
			mit: alle $A_i$ markiert, $B$ nicht markiert)\\
			if (Fall (i)) markiere alle B\\
			else (Fall (ii)) Rückgabe "`unerfüllbar, STOP\\
			end
		\end{enumerate}
		end\\
		Rückgabe "`erfüllbar"'\\
		STOP
	\end{enumerate}
\end{frame}

\begin{frame}{Beispiel: Markierungsalgorithmus}
	$$F=(\neg A \lor \neg B \lor \neg D) \land \neg E \land (\neg C \lor A) \land C \land B \land (\neg G \lor D) \land G$$
\end{frame}

\note{\scriptsize
	\begin{align*}
		F &= (\neg A \lor \neg B \lor \neg D) \land \neg E \land (\neg C \lor A) \land C \land B \land (\neg G \lor D) \land G\\
		&= (A \land B \land D \rightarrow 0) \land (E \rightarrow 0) \land (C \rightarrow A) \land \\
		& \qquad (1 \rightarrow C) \land (1 \rightarrow B) \land (G \rightarrow D) \land (1 \rightarrow G) \\
		&= (A \land \stkout{B} \land D \rightarrow 0) \land (E \rightarrow 0) \land (\stkout{C} \rightarrow A) \land \\
		& \qquad \stkout{(1 \rightarrow C)} \land \stkout{(1 \rightarrow B)} \land (\stkout{G} \rightarrow D) \land \stkout{(1 \rightarrow G)} \\
		&= (\stkout{A} \land \stkout{B} \land \stkout{D} \rightarrow 0) \land (E \rightarrow 0) \land \stkout{(\stkout{C} \rightarrow A)} \land \\
		& \qquad \stkout{(1 \rightarrow C)} \land \stkout{(1 \rightarrow B)} \land \stkout{(\stkout{G} \rightarrow D)} \land \stkout{(1 \rightarrow G)}\\
		& \textrm{unerfüllbar, wegen } (\stkout{A} \land \stkout{B} \land \stkout{D} \rightarrow 0)
	\end{align*}
}
\begin{frame}{Erfüllbarkeitstest (Hornklauselalgorithmus)}
	Sei $S$ Menge von Hornklauseln\\
	while ($S$ enthält eine positive Klausel $\left\{A_i\right\}$)\\
	\qquad entferne alle Klauseln, die $A_i$ enhalten\\
	\qquad und entferne $\overline{A_i}$ aus den verbliebenen Klauseln\\
	end\\
	if ($S$ enhält die leere Klausel $\Box$)\\
	\qquad Rückgabe "`unerfüllbar"'\\
	else Rückgabe "`erfüllbar"'\\
	end
\end{frame}

\note{\begin{align*}
F &= \left\{\left\{\overline{A}, \overline{B}, \overline{D}\right\}, \left\{\overline{E}\right\}, \left\{\overline{C}, A\right\}, \left\{C\right\}, \left\{B\right\}, \left\{\overline{G}, D\right\}, \left\{G\right\}\right\}\\
 &\equiv \left\{\left\{\overline{A}, \overline{B}, \overline{D}\right\}, \left\{\overline{E}\right\}, \left\{\stkout{\overline{C}}, A\right\}, \stkout{\left\{C\right\}}, \left\{B\right\}, \left\{\overline{G}, D\right\}, \left\{G\right\}\right\}\\
 &\equiv \left\{\left\{\overline{A}, \stkout{\overline{B}}, \overline{D}\right\}, \left\{\overline{E}\right\}, \left\{\stkout{\overline{C}}, A\right\}, \stkout{\left\{C\right\}}, \stkout{\left\{B\right\}}, \left\{\overline{G}, D\right\}, \left\{G\right\}\right\}\\
 &\equiv \left\{\left\{\stkout{\overline{A}}, \stkout{\overline{B}}, \overline{D}\right\}, \left\{\overline{E}\right\}, \stkout{\left\{\stkout{\overline{C}}, A\right\}}, \stkout{\left\{C\right\}}, \stkout{\left\{B\right\}}, \left\{\overline{G}, D\right\}, \left\{G\right\}\right\}\\
 &\equiv \left\{\left\{\stkout{\overline{A}}, \stkout{\overline{B}}, \overline{D}\right\}, \left\{\overline{E}\right\}, \stkout{\left\{\stkout{\overline{C}}, A\right\}}, \stkout{\left\{C\right\}}, \stkout{\left\{B\right\}}, \left\{\stkout{\overline{G}}, D\right\}, \stkout{\left\{G\right\}}\right\}\\
 &\equiv \{\underbrace{\left\{\stkout{\overline{A}}, \stkout{\overline{B}}, \stkout{\overline{D}}\right\}}_\Box, \left\{\overline{E}\right\}, \stkout{\left\{\stkout{\overline{C}}, A\right\}}, \stkout{\left\{C\right\}}, \stkout{\left\{B\right\}}, \stkout{\left\{\stkout{\overline{G}}, D\right\}}, \stkout{\left\{G\right\}}\}\\
\end{align*}

}

\begin{frame}{Erfüllbarkeit allgemeiner Klauselmengen: Davis-Putnam-Regeln}
	\begin{itemize}
		\item Sei $S=\left\{K_1, \ldots, K_k\right\}$ Klauselmenge, $L$ ein Literal, $\overline{L}$ das negierte Literal. Definiere:
		\begin{itemize}
			\item $S_L^+=\left\{K_j \in S \mid L \in K_j\right\}, S_L^-=\left\{K_j \in S \mid \overline{L} \in K_j\right\}, S_L^0=\left\{K_j \in S \mid L, \overline{L} \notin K_j\right\}$
			\item $POS_L(S)=S_L^0 \bigcup \left\{K_j \setminus \left\{L\right\} \mid K_j \in S_L^+\right\}; NEG_L(S)=S_L^0 \bigcup \left\{K_j \setminus \left\{\overline{L}\right\} \mid K_j \in S_L^-\right\}$
		\end{itemize}
		\item Es gilt: Die Klauselmenge $S$ ist erfüllbar gdw. wenigstens eine der beiden Klauselmengen $POS_L(S)$ und $NEG_L(S)$ erfüllbar ist
		\begin{itemize}
			\item Vorteil: Anzahl der Literale wird schrittweise reduziert
			\item Nachteil: in jedem Schritt verdoppelt sich i.A. die Zahl der Klauselmengen (vielfach aber auch nicht, s.u.)
		\end{itemize}
		\item Spezialfälle
		\begin{itemize}
			\item $S_L^-=\varnothing$: S erfüllbar gdw. $NEG_L(S)=S_L^0$ erfüllbar (analog für $S_L^+=\varnothing$)
			\item Unit-Regel: falls $\left\{L\right\} \in S$, dann $S$ erfüllbar gdw. $NEG_L(S)$ erfüllbar
		\end{itemize}
	\end{itemize}
\end{frame}

\begin{frame}{Resolution}
	\begin{itemize}
		\item Beweis der Unerfüllbarkeit allgemeiner Klauselmengen durch Herleitung der leeren Klausel
		\item Definition: Resolvente
		\begin{itemize}
			\item sei $L$ ein Literal und $\overline{L}$ das negierte Literal
			\item sei $K_1$ eine Klausel, die $L$ enthält, $K_2$ eine Klausel, die $\overline{L}$ enthält
			\item dann heißt die Klausel $RES_L(K_1, K_2)=\left(K_1 \setminus \left\{L\right\}\right) \cup \left(K_2 \setminus \left\{\overline{L}\right\}\right)$
			\item Gesprochen: "`Resolvente der Klauseln $K_1$ und $K_2$ nach $L$"'
			\item Notation: $\left\{K_1, K_2\right\} \vdash_{res} RES_L$
		\end{itemize}
		\item Resolutionslemma\\
		sei $I$ ein Modell von $K_1$ und $K_2$, dann ist $I$ auch Modell von $RES_L(K_1, K_2)$, d.h. $I \cap K_1 \neq \varnothing \land I \cap K_2 \neq \varnothing \rightarrow I \cap RES_L(K_1, K_2) \neq \varnothing$
	\end{itemize}
\end{frame}

\note{
Lemma: "`Hilfssatz"', auch "`Stichwort"' oder "`Hauptgedanke"'. Wird zum Beweisen von anderen, allgemeingültigeren "`Sätzen"' ("`Theoremen"') verwendet.
}

\begin{frame}{Grundresolutionstheorem}
	\begin{itemize}
		\item Zu $S$ erfüllbarkeitsäquivalente Klauselmenge $RES_L(S)$
		\begin{itemize}
			\item sei $S$ Klauselmenge; $n$ Anzahl der Variablen; definiere:\\
			$RES_L(S)=S_L^0 \bigcup \left\{RES_L(K_1, K_2) \mid K_1 \in S_L^+, K_2 \in S_L^-\right\}$
			\item beachte: $RES_L(S)$ enhält $L$ bzw. $\overline{L}$ nicht mehr
			\item es gilt: $S$ ist erfüllbar gdw. $RES_L(S)$ ist erfüllbar
		\end{itemize}
		\item Grundresolutionstheorem: eine Klauselmenge $S$ ist unerfüllbar gdw. $S$ lässt sich mittels Resolution widerlegen, d.h. die leere Klausel ist herleitbar
		\item Systematische Durchführung:\\
		$S_0=S$, $S_i=RES_{L_i}\left(S_{i-1}\right)$; alle $S_i, i \leq n$ sind erfüllbarkeitsäquivalent\\
		$S_n$ enthält kein Literal mehr, d.h. $S_n=\varnothing$ (erfüllbar) oder $S_n=\left\{\Box\right\}$ (unerfüllbar)
	\end{itemize}
\end{frame}

\begin{frame}{Resolution: Beispiel und Hinweis}
	\begin{itemize}
		\item Ist folgende Formel (un-)erfüllbar?: $(\neg A \lor B) \land (\neg B \lor C) \land A \land \neg C$
		\item \textbf{Vorsicht!} 
		\begin{itemize}
			\item $\left\{A, \neg B\right\}$ und $\left\{\neg A, B\right\}$ haben als Resolvente \textbf{nicht} $\Box$! Es wird immer nur ein (komplementäres) Literal entfernt!
			\item $\left\{B, \neg B\right\}$ führt nicht zu $\Box$! Es werden immer zwei Klauseln benötigt!
		\end{itemize}
		
	\end{itemize}
\end{frame}

\note{
$\left\{\neg A, B\right\}$\\
$\left\{\neg B, C\right\}$\\
$\left\{A\right\}$\\
$\left\{\neg C\right\}$\\
$\left\{B\right\}$\\
$\left\{C\right\}$\\
$\Box$
}