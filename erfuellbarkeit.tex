\section{Erfüllbarkeit aussagenlogischer Formeln}
\begin{frame}{Entscheidungsproblem SAT}
	\begin{itemize}
		\item Ist eine gegebene aussagenlogische Formel erfüllbar / nicht erfüllbar?
		\item Gesucht sind \emph{Algorithmen} (Verfahren, Handlungsanweisungen), die für eine (beliebige) aussagenlogische Formel als Eingabe nach endlich vielen Schritten mit (korrekter) Aussage ja/nein terminieren.
		\item trivialer Algorithmus: Berechnung der Wahrheitstafel $\rightarrow$ nicht effizient
		\item Bemerkung: jede aussagenlogische Formel lässt sich effizient in eine erfüllbarkeitsäquivalente Formel in KNF umschreiben (unter Einführung zusätzlicher Variablen) $\rightarrow$ betrachte im weiteren Formeln in KNF
	\end{itemize}
\end{frame}

\begin{frame}{Klauselmengen}
	\begin{itemize}
		\item Mengennotation für Formeln in KNF\\
		ersetze Klausel $L_{i1} \lor L_{i2} \lor \ldots$ durch $C_i = \left\{L_{i1}, L_{i2}, \ldots\right\}$ und Formel $F$ in KNF durch Klauselmenge $S=\left\{C_1, C_2, \ldots\right\}$
		\item Belegung $I$ lässt sich ebenfalls als Menge darstellen:\\
		z.B. $I(x_1)=w, I(x_2)=f, \ldots$\\
		dann Schreibweise $I=\left\{x_1, \overline{x_2}, \ldots\right\}$
		\item Belegung $I$ ist Modell der Klausel $C_i$ gdw. $I \cap C_i \neq \varnothing$.\\
		Belegung $I$ ist Modell der Klauselmenge $S$ gdw. $I \cap C_i \neq \varnothing$ für alle $C_i \in S$.
	\end{itemize}
\end{frame}

\begin{frame}{Hornformeln}
	\begin{itemize}
		\item Definition: $F$ ist Hornformel gdw. $F$ ist in KNF und jede Klausel enthält höchstens ein positives Literal.
		\item Beispiel: $F_1=(A \lor \neg B) \land (\neg C \lor \neg A \lor D) \land (\neg A \lor \neg B) \land D \land \neg E$
		\item Es gilt: jede Hornformel lässt sich äquivalent in eine Konjunktion von Implikationen ("`Regeln"') umformen
		\item Beispiel: $F_1=(B \rightarrow A) \land (A \land C \rightarrow D) \land (A \land B \rightarrow f) \land (w \rightarrow D) \land (E \rightarrow f)$
		\item Eine Menge von Hornklauseln heißt auch Logikprogramm
		\begin{itemize}
			\item Tatsachenklausel: ein positives und kein negatives Literal
			\item Prozedurklausel (Regel): ein positives und mindestens ein negatives Literal
			\item Zielklausel (Frageklausel): negative Klausel (ohne positives Literal)
		\end{itemize}
	\end{itemize}
\end{frame}