
\documentclass[18pt,aspectratio=169]{beamer}
%\usetheme{kit}
\usepackage{ngerman}

% metropolis theme: https://github.com/matze/mtheme
\usetheme[sectionpage=progressbar,subsectionpage=progressbar,progressbar=frametitle]{metropolis}
\definecolor{myColor}{HTML}{4477AA}
\definecolor{pitchblack}{HTML}{000000}
\setbeamercolor{alerted text}{fg=myColor}
\setbeamercolor{frametitle}{bg=pitchblack, fg=white}

\makeatletter
\setlength{\metropolis@titleseparator@linewidth}{1pt}
\setlength{\metropolis@progressonsectionpage@linewidth}{2pt}
\setlength{\metropolis@progressinheadfoot@linewidth}{2pt}
\makeatother


% uncomment the following line if you want to hide the navigation symbols
%\beamertemplatenavigationsymbolsempty 

\title{Theoretische Informatik I}
\subtitle{Automaten und Formale Sprachen}
\author{Peter Kossek}

\institute{Berufsakademie Sachsen, Staatliche Studienakademie Leipzig} % Deutsch
\date{}

\makeatletter
\@addtoreset{figure}{section}
\@addtoreset{table}{section}
\makeatother

\renewcommand{\thefigure}{\thesection.\arabic{figure}}

\renewcommand{\thetable}{\thesection.\arabic{table}}

\usepackage{nameref}
\makeatletter
\newcommand*{\currentname}{\@currentlabelname}
\makeatother

\usepackage{circuitikz}

\begin{document}

\selectlanguage{ngerman}
%title page
\begin{frame}
	\titlepage
\end{frame}

\begin{frame}
	\frametitle{Danksagung}
	Diese Vorlesung basiert maßgeblich auf dem Script von\\
	Prof. Jochen Kripfganz\\
	Dozent für Theoretische Informatik an der BA Leipzig bis 2021
\end{frame}

\AtBeginSection{
	\begin{frame}
		\sectionpage
		\tableofcontents[sectionstyle=hide/hide,subsectionstyle=show/show/hide]
	\end{frame}
}

%table of contents
\begin{frame}
	\frametitle{Gliederung}
	\tableofcontents
\end{frame}

\frame{
	\frametitle{Sprachen im Kontext der Informatik}
	
	\begin{itemize}
		\item \emph{Sprache:} Menge zulässiger Worte über einem Alphabet
		\item \emph{Programm:} String über einem Alphabet von Eingabezeichen (Quellcode) bzw. binär über 0/1
		\item \emph{Compiler:} 
		\begin{itemize}
			\item prüft die Zulässigkeit dieses Strings als Satz (Wort) einer Sprache
			\item generiert Folge von Maschinenbefehlen
		\end{itemize}
		\item \emph{Lexikalische Analyse:} Übersetzer-Modul ("`Scanner"') fasst Symbolfolgen zu Schlüsselworten ("`Token"') zusammen
		\item \emph{Syntaxanalyse:}
		\begin{itemize}
			\item "`Parser"' prüft, ob Folge von Token zulässig ist
			\item Semantik-Prüfung erfolgt dabei nicht
		\end{itemize}
	\end{itemize}
}

\begin{frame}
	\frametitle{Literatur}
	\begin{itemize}
	  \item Grundlage der Vorlesung
		\begin{itemize}
			\item Schöning, U.: \textit{Logik für Informatiker}
			\item Schöning, U.: \textit{Theoretische Informatik - kurz gefasst}
		\end{itemize}
		\item Weitere Literatur
		\begin{itemize}
			\item Vossen, G.; Witt, K. U.: Grundlagen der Theoretischen
		\end{itemize}
	\end{itemize}
\end{frame}

\section{Aussagenlogik}

\subsection{Aussagenlogische Formeln}

\begin{frame}{Aussagenlogik: Syntax}
	\begin{itemize}
		\item \emph{Aussage:} Satz, der entweder wahr ($w$) oder falsch ($f$) ist; Aussagenvariable $A$; Wahrheitswert $w(A)$
		\item \emph{Syntax:} Induktive Definition korrekt gebildeter aussagenlogischer Formeln F über Variablenmenge $V=\{A, B, \ldots\}$:
		\begin{itemize}
			\item Die Booleschen Wahrheitswerte $w$ und $f$ sind Formeln
			\item Jede Variable aus $V$ ist eine Formel: \emph{Atome}
			\item Negation (NICHT): $\neg F$ ist eine Formel
			\item Konjugation (UND): $(F_1 \land F_2)$ ist eine Formel
			\item Disjunktion (ODER): $(F_1 \lor F_2)$ ist eine Formel
			\item Implikation ("`wenn \ldots, dann"'): $(F_1 \rightarrow F_2)$ ist eine Formel
			\item Äquivalenz ("`genau dann, wenn"'): $(F_1 \leftrightarrow F_2)$ ist eine Formel
			\item andere Verknüpfungen bilden keine Formel
		\end{itemize}
	\end{itemize}
\end{frame}

\begin{frame}{Präzedenzregeln}
	Vereinbarung zur Reduzierung von Klammern
	\begin{itemize}
		\item Bindung (analog "`Punkt vor Strich"'-Rechnung)
		\begin{itemize}
			\item $\neg$ bindet stärker als $\land$
			\item $\land$ bindet stärker als $\lor$
			\item $\lor$ bindet stärker als $\rightarrow$ und $\leftrightarrow$
		\end{itemize}
		\item Operatoren gleicher Stärke: Auswertung linksassoziativ; z.B. \\ $(A \lor B \lor C)$ steht für $((A \lor B) \lor C)$
		\item äußere Klammer weglassen: $(((A \lor B) \rightarrow C) \land B) \mapsto (A \lor B \rightarrow C) \land B$
	\end{itemize}
\end{frame}

\begin{frame}{Semantik}
	aussagenlogischen Formeln wird eine Bedeutung zugeordnet
	\begin{itemize}
		\item Def. Belegung (Interpretation) $I: V \rightarrow \{f, w\}$\\ den Atomen wird jeweils ein konkreter Wahrheitswert zugeordnet
		\item schrittweise (über die Bewertung von Teilformeln) lassen sich dann zusammengesetzte Formeln bewertn:
		\begin{itemize}
			\item $I(\neg F)=w$ falls $I(F)=f$, sonst $I(\neg F)=f$
			\item $I(F \land G)=w$ falls $I(F)=w$ und $I(G)=w$, sonst $I(F \land G)=f$
			\item $I(F \lor G)=w$ falls $I(F)=w$ oder $I(G)=w$, sonst $I(F \lor G)=f$
			\item $I(F \rightarrow G)=w$ falls $I(F)=f$ oder $I(G)=w$, sonst $I(F \rightarrow G)=f$
			\item $I(F \leftrightarrow G)=w$ falls $I(F)=I(G)$, sonst $I(F \leftrightarrow G)=f$
		\end{itemize}
	\end{itemize}
\end{frame}

\begin{frame}{Semantik: Darstellung per Wahrheitstafel}
	Wahrheitstafel enthält zeilenweise alle möglichen Belegungen
	\begin{table}
		\centering
			\begin{tabular}{|c|c|c|c|c|c|c|}
				\hline
				$A$ & $B$ & $\neg A$ & $A \land B$ & $A \lor B$ & $A \rightarrow B$ & $A \leftrightarrow B$ \\
				\hline
				f & f & w & f & f & w & w \\
				\hline
				f & w & w & f & w & w & f \\
				\hline
				w & f & f & f & w & f & f \\
				\hline
				w & w & f & w & w & w & w \\
				\hline
			\end{tabular}
			\caption{Wahrheitstafel der Grundoperationen}
		\end{table}
		logische Äquivalenz $F_1 \equiv F_2$: $F_1$ und $F_2$ haben gleichen Wahrheitswerteverlauf; z.B.\\
		$A \rightarrow B \equiv \neg A \lor B$\\
		$A \leftrightarrow B \equiv (A \rightarrow B) \land (B \rightarrow A)$
\end{frame}

\begin{frame}{Erfüllbarkeit, Tautologie, Kontradiktion}
	\begin{itemize}
		\item Sei $F$ eine Formel und $I$ eine Belegung. Falls $I$ für alle in $F$ vorkommenden Atome definiert ist, so heißt $I$ \underline{zu $F$ passend}
		\item Falls $I$ zu $F$ passend ist, und es gilt $I(F)=w$, dann heißt $I$ \underline{Modell für $F$}; Schreibweise: $I \models F$
		\item $F$ heißt \underline{erfüllbar}, falls $F$ mindestens ein Modell besitzt; sonst heißt $F$ \underline{unerfüllbar} (Kontradiktion; Schreibweise $F\bot$)
		\item $F$ heißt gültig bzw. \underline{Tautologie}, falls jede zu $F$ passende Belegung $I$ ein Modell für $F$ ist; Schreibweise $\models F$ oder $F\top$ \\
		Beispiel: $(A \land B) \rightarrow A$
		\item es gilt: $F$ ist Tautologie gdw. $\neg F$ ist Kontradiktion
	\end{itemize}
\end{frame}

\begin{frame}{Äquivalenz aussagenlogischer Formeln}
	\begin{itemize}
		\item Zwei Formeln $F$ und $G$ heißen äquivalent ($F \equiv G$), falls für alle Belegungen $I$ gilt: $I(F)=I(G)$
		\item Äquivalenz kann bewiesen (oder widerlegt) werden durch aufstellen der jeweiligen Wahrheitstafeln
		\item Beispiele
		\begin{itemize}
			\item $A \rightarrow B \equiv \neg A \lor B$
			\item $A \leftrightarrow B \equiv (A \land B) \lor (\neg A \land \neg B) \equiv (A \rightarrow B) \land (B \rightarrow A)$
		\end{itemize}
		\item Ersetzbarkeitstheorem:\\
		enthält eine Formel $F$ eine Teilformel $G$ und wird $G$ durch eine äquivalente Formel ersetzt, so entsteht eine zu $F$ äquivalente Formel
	\end{itemize}
\end{frame}

\begin{frame}{Äquivalenzen als Rechenregeln}
	\begin{itemize}
		\item Idempotenz
		\begin{itemize}
			\item $(A \land A) \equiv A$
			\item $(A \lor A) \equiv A$
		\end{itemize}
		\item Kommutativgesetz
		\begin{itemize}
			\item $(A \land B) \equiv (B \land A)$
			\item $(A \lor B) \equiv (B \lor A)$
		\end{itemize}
		\item Assoziativgesetz
		\begin{itemize}
			\item $(A \land (B \land C)) \equiv ((A \land B) \land C)$
			\item $(A \lor (B \lor C)) \equiv ((A \lor B) \lor C)$
		\end{itemize}
		\item Distributivgesetz
		\begin{itemize}
			\item $(A \land (B \lor C)) \equiv ((A \land B) \lor (A \land C))$
			\item $(A \lor (B \land C)) \equiv ((A \lor B) \land (A \lor C))$
		\end{itemize}
	\end{itemize}
\end{frame}

\begin{frame}{Äquivalenzen als Rechenregeln (Fortsetzung)}
	\begin{itemize}
		\item Absorptionsgesetz
		\begin{itemize}
			\item $(A \land (A \lor B)) \equiv A$
			\item $(A \lor (A \land B)) \equiv A$
		\end{itemize}
		\item Doppelnegation
		\begin{itemize}
			\item $\neg \neg A \equiv A$
		\end{itemize}
		\item deMorgansche Regeln
		\begin{itemize}
			\item $\neg (A \land B) \equiv (\neg A \lor \neg B)$
			\item $\neg (A \lor B) \equiv (\neg A \land \neg B)$
		\end{itemize}
		\item Tautologieregeln: sei $A$ Tautologie; $A\top$ bzw. $\models A$
		\begin{itemize}
			\item $(A \lor B) \equiv A$; $(A \land B) \equiv B$
		\end{itemize}
		\item Kontradiktionsregeln: sei $A$ Kontradiktion; $A\bot$
		\begin{itemize}
			\item $(A \lor B) \equiv B$; $(A \land B) \equiv A$
		\end{itemize}
	\end{itemize}
\end{frame}

\begin{frame}{Beispiel: direkter Beweis der Gültigkeit einer Formel (als Alternative zur Wahrheitstafel)}
	\begin{columns}
		\column{.5\textwidth}
		\begin{eqnarray*}
				F &=& (A \land (A \rightarrow B)) \rightarrow B\\
				  &\equiv& \neg (A \land (A \rightarrow B)) \lor B\\
				  &\equiv& \neg (A \land (\neg A \lor B)) \lor B\\
				  &\equiv& \neg A \lor \neg(\neg A \lor B) \lor B\\
				  &\equiv& \neg A \lor (A \land \neg B) \lor B\\
				  &\equiv& \neg A \lor (A \lor B) \land (\neg B \lor B)\\
				  &\equiv& \neg A \lor (A \lor B) \land w\\
				  &\equiv& \neg A \lor A \lor B\\
				  &\equiv& w \lor B\\
				  &\equiv& w
		\end{eqnarray*}
		\column{.5\textwidth}
		\begin{table}
			\begin{tabular}{|c|c|c|c|c|}
				\hline
				$A$ & $B$ & $A \rightarrow B$ & $A \land (A \rightarrow B)$ & $F$ \\
				\hline
				f & f & w & f & w \\
				\hline
				f & w & w & f & w \\
				\hline
				w & f & f & f & w \\
				\hline
				w & w & w & w & w \\
				\hline
			\end{tabular}
			\caption{Wahrheitstafel}
			\label{tblWahrheitswerteGrundoperationen}
		\end{table}
	\end{columns}		
\end{frame}

\begin{frame}{Normalformen}
	\begin{itemize}
		\item \emph{Literal:} Aussagenvariable $A$ ("`positives Literal"') oder negierte Aussagenvariable $\neg A$ ("`negatives Literal"')
		\item \emph{Negationsnormalform (NNF)}: Formel $F$ ist in NNF, falls $\rightarrow$ und $\leftrightarrow$ aufgelöst sind und alle Negationszeichen $\neg$ unmittelbar vor einer Aussagenvariable stehen.\\
		beachte: NNF ist nicht eindeutig
		\item \emph{Monom:} Konjunktion von Literalen, d.h. Formel der Art\\
		$\bigwedge_iL_i=L_1\land L_2\land\ldots\land L_k$
		\item \emph{Klausel:} Disjunktion von Literalen, d.h. Formel der Art\\
		$\bigvee_iL_i=L_1\lor L_2\lor\ldots\lor L_k$
	\end{itemize}
\end{frame}

\begin{frame}{Disjunktive Normalform (DNF) und Konjunktive Normalform (KNF)}
	\begin{itemize}
		\item Disjunktive Normalform \emph{DNF:} Disjunktion von Konjunktionen (Monomen)\\
		$F=\bigvee_i\left(\bigwedge_jL_{ij}\right)$
		\item Konjunktive Normalform \emph{KNF:} Konjunktion von Disjunktionen (Klauseln)\\
		$F=\bigwedge_i\left(\bigvee_jL_{ij}\right)$
		\item Für jede Formel existiert eine äquivalente Formel in KNF und eine äquivalente Formel in DNF
		\begin{itemize}
			\item Erzeuge NNF $\mapsto$ DNF bzw. KNF\\
			DNF: ersetze Ausdrücke der Art $(F \land (G \lor H))$ durch $(F \land G) \lor (F \land H)$ solange solche Teilformen existieren\\
			KNF: ersetze Ausdrücke der Art $(F \lor (G \land H))$ durch $(F \lor G) \land (F \lor H)$ solange solche Teilformeln existieren
		\end{itemize}
	\end{itemize}
\end{frame}

\begin{frame}{Konstruktion kanonische DNF/KNF aus Wahrheitstafel}
	\begin{itemize}
		\item \emph{Kanonische} DNF bzw. KNF: jede Aussagenvariable kommt in jedem Monom bzw. jeder Klausel genau einmal vor (positiv oder negativ)
		\item Kanonische NF lassen sich aus der Wahrheitstafel der Formel ablesen
		\begin{itemize}
			\item Kanonische DNF (KDNF)
			\begin{itemize}
				\item Je ein Monom für jede Belegung mit Wahrheitswert $w$
				\item Setze $L_j=A_j$, falls $w(A_j)=w$, $L_j=\neg A_j$ sonst
			\end{itemize}
			\item Kanonische KNF (KKNF)
			\begin{itemize}
				\item Je eine Klausel für jede Belegung mit Wahrheitswert $f$
				\item Setze $L_j=\neg A_j$, falls $w(A_j)=w$, $L_j=A_j$ sonst
			\end{itemize}
		\end{itemize}
		\item Übungsbeispiel (KNF, aber nicht kanonisch)\\
		$F=\neg B \land (A \lor \neg C)$
	\end{itemize}
\end{frame}

\begin{frame}{Beispiel: Konstruktion von KDNF und KKNF aus Wahrheitstafel}
	\begin{columns}
		\column{.3\textwidth}
		\begin{center}
			\begin{tabular}{|c|c|c||c|}
				\hline
				A & B & C & F \\
				\hline
				0 & 0 & 0 & 1 \\
				0 & 0 & 1 & 0 \\
				0 & 1 & 0 & 0 \\
				0 & 1 & 1 & 0 \\
				1 & 0 & 0 & 1 \\
				1 & 0 & 1 & 1 \\
				1 & 1 & 0 & 0 \\
				1 & 1 & 1 & 0 \\
				\hline
			\end{tabular}
		\end{center}
		\column{.7\textwidth}
		\begin{itemize}
			\item [KDNF:] Werte für F=1 auslesen
			$$(\neg A \land \neg B \land \neg C) \lor (A \land \neg B \land \neg C) \lor (A \land \neg B \land C)$$
			\item [KKNF:] Werte für F=0 auslesen und Literale negieren
			\begin{align*}
				&(A \lor B \lor \neg C) \land (A \lor \neg B \lor C) \land \\
				&(A \lor \neg B \lor \neg C) \land (\neg A \lor \neg B \lor C) \land (\neg A \lor \neg B \lor \neg C)
			\end{align*}
			\item Minimierung der KKNF und KDNF durch Karnaugh-Veitch-Diagramm möglich (dann aber i.d.R. nicht mehr kanonisch)
		\end{itemize}
	\end{columns}
\end{frame}

\begin{frame}{Aussagenlogische Formeln in Java}
	\begin{itemize}
		\item Datentyp \texttt{boolean}\\
		\texttt{p,q; //zulässige Werte: true, false}
		\item Logische Verküpfungen\\
		\begin{tabbing}
			UND: \quad \= \texttt{\&\&}\\
			ODER: \> \texttt{||}\\
			NICHT: \> \texttt{!}
		\end{tabbing}
		\item Beispiel \\
			\texttt{boolean p,q;}\\
			\texttt{int x = 8;}\\
			\texttt{p = false;}\\
			\texttt{q = x == 10; // false}\\
			\texttt{p = (p || q) \&\& x < 10; //  false}
		\item Vorrangregeln: \texttt{!} bindet stärker als \texttt{\&\&} bindet stärker als \texttt{||}
	\end{itemize}
\end{frame}

\subsection{Mengen/Relationen}

\begin{frame}{Exkurs: Mengen}
	\begin{itemize}
		\item \emph{Menge:} Zusammenfassung unterschiedlicher Objekte (Elemente) zu einer Einheit
		\item \emph{Bezeichner:} $M=\left\{x \in G \mid \textrm{Bedingung}_1, \textrm{Bedingung}_2, \ldots \right\}$; $G$: Grundmenge
		\item Beispiel: $M=\left\{n \in \mathbb{N} \mid n^2-3n+2=0 \right\}=\left\{1,2\right\}$
		\item \emph{Leere Menge:} $\varnothing=\left\{{}\right\}$
		\item \emph{Teilmengen:} $A \subseteq B$ gdw. $x \in A \rightarrow x \in B$ für alle $x \in G$
		\item \emph{Gleichheit von Mengen:} $A=B$ gdw. $A \subseteq B \land B \subseteq A$\\
			z.B. $A = \left\{a, b, a\right\}, B = \left\{b, a\right\}$ es gilt: $A = B$
		\item \emph{Mächtigkeit (endlicher) Mengen:}\\
		$|M|=$ Anzahl verschiedener Elemente von $M$; z.B. $|\left\{a, b, a\right\}|=2$
	\end{itemize}
\end{frame}

\begin{frame}{Mengenverknüpfungen}
	\begin{itemize}
		\item Durchschnitt: $A \cap B = \left\{x \mid x \in A \land x \in B\right\}$
		\item Vereinigung: $A \cup B = \left\{x \mid x \in A \lor x \in B\right\}$
		\item Differenz: $A \setminus B = \left\{x \mid x \in A \land x \notin B\right\}$\\
			insbesondere: Komplement $A'= \bar{A} = G \setminus A$
		\item es gelten Kommutativ-, Assoziativ- und Distributivgesetze, insbesondere\\
			$A \cap \left(B \cup C\right) = \left(A \cap B\right) \cup \left(A \cap C\right)$\\
			$A \cup \left(B \cap C\right) = \left(A \cup B\right) \cap \left(A \cup C\right)$
		\item Potenzmenge $P(M) =$ Menge aller Teilmengen von M, z.B.\\
			$M=\left\{a, b, c\right\}$\\
			$P(M)=\left\{\varnothing, \{a\}, \{b\}, \{c\}, \{a,b\}, \{a,c\}, \{b,c\}, \{a,b,c\}\right\}$\\
			es gilt: $|P(M)|=2^{|M|}$
	\end{itemize}
\end{frame}

\begin{frame}{Relationen}
	Begriffsbestimmungen
	\begin{itemize}
		\item Kartesisches Produkt zweier Mengen $A, B$\\
			Menge aller geordneten Paare $(a,b): A \times B = \left\{(a,b) \mid a \in A, b \in B\right\}$
		\item Binäre Relation $R$ zwischen $A$ und $B$\\
			$R \subseteq A \times B$, Sprechweise: $(a,b) \in R \Rightarrow a$ und $b$ stehen in Relation $R$
		\item Darstellungen: Tabelle, Pfeildiagramm
		\item Produkt (Komposition) $R_1 \circ R_2$\\
			sei $R_1 \subseteq A \times B, R_2 \subseteq B \times C$\\
			$R_1 \circ R_2 = \left\{(a,c) \mid \textrm{es ex. ein } b \in B \textrm{ mit } (a,b) \in R_1, (b,c) \in R_2\right\}$
		\item Inverse Relation $R^{-1}$\\
			sei $R \subseteq A \times B \Rightarrow R^{-1}\subseteq B \times A$\\
			$R^{-1}=(b,a) \mid a \in A, b \in B, (a,b) \in R$
	\end{itemize}
\end{frame}

\begin{frame}{Binäre Relationen \emph{in einer} Menge $M: R \subset M \times M$}
	\begin{itemize}
		\item Bezeichnung: es gelte $a,b,c \in M$
		\item \emph{Identität $I$:} $I=\left\{(a,a) \mid a \in M\right\}$
		\item $R$ heißt \emph{reflexiv} gdw. $I \subseteq R$
		\item $R$ heißt \emph{symmetrisch} gdw. $(a,b) \in R \rightarrow (b,a) \in R$
		\item $R$ heißt \emph{asymmetrisch} gdw. $(a,b) \in R \rightarrow (b,a) \notin R$
		\item $R$ heißt \emph{antisymmetrisch} gdw. $(a,b) \in R \land (b,a) \in R \rightarrow a = b$
		\item $R$ heißt \emph{transitiv} gdw. $((a,b) \in R \land (b,c) \in R) \rightarrow (a,c) \in R$
		\item \emph{Transitive Hülle} $T$ von $R$\\
			$T=\{(a,b) \mid (a,b) \in R, \textrm{ oder: es gibt } c_0, c_1, \ldots, c_m \in M$\\
			$\textrm{mit } (a,c_0) \in R, (c_0, c_1) \in R, \ldots, (c_m,b) \in R\}$
	\end{itemize}
\end{frame}

\begin{frame}{Äquivalenzrelationen}
	\begin{itemize}
		\item Definition: $R$ in $M$ heißt Äquivalenzrelation in $M$ gdw. $R$ ist reflexiv, symmetrisch und transitiv
		\item Äquivalenzklassen: $R[x]=\{y \mid (x,y) \in R\}$ heißt Äquivalenzklasse von $x$ bezüglich $R$;\\
			es gilt: zwei Äquivalenzklassen $R[x], R[y]$ sind entweder identisch oder disjunkt
		\item Beispiel (Zahlentheorie): Kongruenz\\
			zwei ganze Zahlen $a,b$ heißen kongruent bzgl. $m \in \mathbb{N}$ gdw. sie haben bei Division durch $m$ den gleichen Rest\\
			Schreibweise: $a \equiv b\ (\mathrm{mod}\ m)$\\
			Rechenregeln: sei $a \equiv a'\ (\mathrm{mod}\ m), b \equiv b'\ (\mathrm{mod}\ m)$\\
			$a+b\equiv a'+b'\ (\mathrm{mod}\ m), a \cdot b \equiv a' \cdot b'\ (\mathrm{mod}\ m)$
	\end{itemize}
\end{frame}

\begin{frame}{Abbildungen}
	\begin{itemize}
		\item Sei $R$ eine Relation zwischen den Mengen $A,B$
		\begin{itemize}
			\item $R$ heißt \emph{linkseindeutig} gdw. $(a_1,b)\in R\land (a_2,b)\in R \rightarrow a_1=a_2$
			\item $R$ heißt \emph{rechtseindeutig} gdw. $(a,b_1)\in R\land (a,b_2)\in R \rightarrow b_1=b_2$
		\end{itemize}
		\item Eine \emph{Abbildung} $f$ von $A$ \emph{in} die Menge $B$ entspricht einer rechtseindeutigen Relation $F$ zwischen $A$ und $B$, für die es zu jedem $a\in A$ ein $b\in B$ gibt mit $(a,b) \in F$\\
		Bezeichnung: $f: A \rightarrow B$ bzw. $b=f(a)$
		\item Eine Abbildung $f$ heißt \emph{surjektiv} (Abbildung \emph{auf} $B$), wenn es zu jedem $b\in B$ ein $a\in A$ gibt mit $b=f(a)$.\\
		Schreibweise: $B = f(A)$
		\item Eine Abbildung $f$ heißt \emph{injektiv} (eineindeutig), wenn die zugehörige Relation $F$ auch linkseindeutig ist
		\item Eine Abbildung $f$ heißt \emph{bijektiv}, wenn sie surjektiv und injektiv ist. Die inverse Relation $F^{-1}$ erklärt dann die inverse Abbildung $a = f^{-1}(b)$ bzw. $A=f^{-1}(B)$
	\end{itemize}
\end{frame}

\subsection{Boolesche Rechenregeln}

\begin{frame}{Boolesche Algebra: Rechenregeln}
	\begin{columns}
		\column{.5\textwidth}
		\emph{Abstrakte Definition}\\
		Boolesche Algebra $(B, +, \times, \kappa)$
		\begin{itemize}
			\item $B$: Grundmenge, enthält insbesondere 0 und 1
			\item $+$ und $\times$: zweistellige Operationen auf $B$
			\item $\kappa$: einstellige Operation auf $B$ (Komplement)
			\item dazu gelten die folgenden vier Rechengesetze $(a,b,c\in B)$
		\end{itemize}
		\column{.5\textwidth}
		\emph{vorausgesetzte Rechenregeln}\\
		\begin{itemize}
			\item Kommutativgesetze\\
				$a+b=b+a$, $a \times b=b \times a$
			\item Distributivgesetze\\
				$a \times (b + c) = (a \times b) + (a\times c)$\\
				$a + (b \times c) = (a + b) \times (a + c)$\\
			\item Neutralitätsgesetze\\
				$a \times 1 = a$, $a + 0 = a$
			\item Komplementgesetze
				$a + \kappa(a) = 1$, $a \times \kappa(a) = 0$
		\end{itemize}
	\end{columns}		
\end{frame}

\begin{frame}{Beispiele}
	\begin{itemize}
		\item Potenzmenge; $B=P(M)$\\
			$+:\cup$, $\times:\cap$, $\kappa(A):M\setminus A$, $0:\varnothing$, $1:M$
		\item $B$: Menge der n-stelligen aussagenlogischen Formeln\\
			(gebildet aus maximal n verschiedenen Aussagenvariablen, zzgl. $w$, $f$)\\
			$+:\lor$, $\times:\land$, $\kappa:\neg$, $0:f$, $1:w$
		\item Schaltfunktionen-Algebra
	\end{itemize}
\end{frame}

\begin{frame}{Abgeleitete Rechenregeln für Boolesche Ausdrücke}
	\begin{itemize}
		\item Dualitätsprinzip der Booleschen Algebra:\\
			zu jeder Rechenregel existiert eine zweite (duale) Rechenregel, die durch Vertauschung $+$ mit $\times$ und $0$ mit $1$ entsteht
		\item Regeln\\
		\begin{tabbing}
		\hspace{12em}\=\hspace{8em}\=\kill
		Idempotenz:	\> $a+a=a$  \> $a\times a = a$\\ 
		Dominanzgesetz:	\> $a+1=a$ \> $a\times 0 = 0$\\ 
		Absorptionsgesetz:	\> $a+(a\times b) = a$ \> $a \times (a + b) = a$\\ 
		Gleichungsvereinfachung:	\> aus $b+a=c+a$ und $b+\kappa(a)=c+\kappa(a)$ \\ 
									\> folgt $b=c$ \\ 
		Assoziativgesetz:	\> $a+(b+c)=(a+b)+c$ \\ 
							\> $a\times (b\times c)=(a\times b)\times c$ \\ 
		De Morgansche Gesetze:	\> $\kappa(a+b) = \kappa(a) \times \kappa(b)$ \\ 
								\> $\kappa(a\times b) = \kappa(a) + \kappa(b)$   
		\end{tabbing} 
	\end{itemize}
\end{frame}

\begin{frame}{Beweisbeispiele}
	\begin{columns}
		\column{.3\textwidth}
		\begin{align*}
			a+a&=a\\
			a+a&=(a+a)\times 1\\
			   &=(a+a)\times (a+\kappa(a))\\
			   &=a+(a\times \kappa(a))\\
			   &=a+0\\
			   &=a
		\end{align*}
		\column{.3\textwidth}
		\begin{align*}
			a+1&=1\\
			a+1&=(a+1)\times 1\\
			&=(a+1)\times (a+\kappa(a))\\
			&=a+(1\times \kappa(a))\\
			&=a+\kappa(a)\\
			&=1
		\end{align*}
		\column{.3\textwidth}
		\begin{align*}
			a+(a\times b)&=a\\
			a+(a\times b)&=(a\times 1)+(a\times b)\\
			&=a\times (1+b)\\
			&=a\times 1\\
			&=a\\
		\end{align*}
	\end{columns}	
\end{frame}

\begin{frame}{Beispiel: Schaltfunktionen-Algebra}
	\begin{itemize}
		\item Jeder n-stellige Boolesche Ausdruck stellt eine n-stellige Schaltfunktion $f(x_1,\ldots,x_n)$, d.h. $f:\left\{0,1\right\}^n\rightarrow\left\{0,1\right\}$ dar
		\item Bezeichnungen an Stelle $f,w,\lor,\land,\neg:0,1,+,\cdot,\textrm{Überstrich}$
		\item Schaltfunktionen-Algebra:\\
		$B=$ Menge aller n-stelligen Schaltfunktionen\\
		$0:f\equiv 0$\\
		$1:f\equiv 1$\\
		$+:\mathrm{Max}[f,g](x_1,\ldots,x_n)=\mathrm{max}\left\{f(x_1,\ldots,x_n),g(x_1,\ldots,x_n)\right\}$ (zeilenweise)\\
		$\cdot:\mathrm{Min}[f,g](x_1,\ldots,x_n)=\mathrm{min}\left\{f(x_1,\ldots,x_n),g(x_1,\ldots,x_n)\right\}$ (zeilenweise)\\
		$\kappa : \kappa [f](x_1,\ldots,x_n)=1-f(x_1,\ldots,x_n)$
	\end{itemize}
\end{frame}

\ctikzset{
	logic ports=european,
	tripoles/european not symbol=ieee circle}

\begin{frame}[shrink=10]{Gatter IEC 60617-12}
	\centering
	\begin{tabular}{llc}
	AND		& $Y=A\cdot B=AB$ & \begin{circuitikz}\draw	(0,0) node[and port] (gate) {} (gate.in 1) node[left](i1) {A}	(gate.in 2) node[left](i2) {B} (gate.out) node[right](o) {Y};	\end{circuitikz} \\
	OR		& $Y=A+B$ & \begin{circuitikz}\draw	(0,0) node[or port] (gate) {} (gate.in 1) node[left](i1) {A}	(gate.in 2) node[left](i2) {B} (gate.out) node[right](o) {Y};	\end{circuitikz} \\
	NOT		& $Y=\overline{A}$ & \begin{circuitikz}\draw	(0,0) node[not port] (gate) {} (gate.in 1) node[left](i1) {A} (gate.out) node[right](o) {Y};	\end{circuitikz} \\
	NAND	& $Y=\overline{AB}$ & \begin{circuitikz}\draw	(0,0) node[nand port] (gate) {} (gate.in 1) node[left](i1) {A}	(gate.in 2) node[left](i2) {B} (gate.out) node[right](o) {Y};	\end{circuitikz} \\
	NOR		& $Y=\overline{A+B}$ & \begin{circuitikz}\draw	(0,0) node[nor port] (gate) {} (gate.in 1) node[left](i1) {A}	(gate.in 2) node[left](i2) {B} (gate.out) node[right](o) {Y};	\end{circuitikz} \\
	XOR		& $Y=A\overline{B}+\overline{A}B$ & \begin{circuitikz}\draw	(0,0) node[xor port] (gate) {} (gate.in 1) node[left](i1) {A}	(gate.in 2) node[left](i2) {B} (gate.out) node[right](o) {Y};	\end{circuitikz} \\
	XNOR	& $Y=(\overline{A}+B)(A+\overline{B})$ & \begin{circuitikz}\draw	(0,0) node[xnor port] (gate) {} (gate.in 1) node[left](i1) {A}	(gate.in 2) node[left](i2) {B} (gate.out) node[right](o) {Y};	\end{circuitikz} \\
	\end{tabular}
\end{frame}

\begin{frame}{Beispiele}
	\centering
	$y=((x_1\cdot x_2)\cdot(x_2+x_3))\cdot \overline{(x_2 + x_3)}$\\
	\vspace{1em}
	\begin{circuitikz}
		\draw
		(0,0) coordinate[label={[below]$x_1$}](x1)
		(0.5,0) coordinate[label={[below]$x_2$}](x2)
		(1,0) coordinate[label={[below]$x_3$}](x3)
		
		(0,6) coordinate(x1End)
		(.5,6) coordinate(x2End)
		(1,6) coordinate(x3End)
		
		(x1) -| (x1End)
		(x2) -| (x2End)
		(x3) -| (x3End)
		
		(3,1) node[and port] (Aandx1x2) {}
		(3,3) node[or port] (Borx2x3) {}
		(5,2) node[and port] (AandB) {}
		(3,5) node[nor port] (nor) {}
		(7,4) node[and port] (andAll) {}
		
		(Aandx1x2.in 1) to[short, -*] (x1 |- Aandx1x2.in 1)
		(Aandx1x2.in 2) to[short, -*] (x2 |- Aandx1x2.in 2)
		
		(Borx2x3.in 1) to[short, -*] (x2 |- Borx2x3.in 1)
		(Borx2x3.in 2) to[short, -*] (x3 |- Borx2x3.in 2)
		
		(nor.in 1) to[short, -*] (x2 |- nor.in 1)
		(nor.in 2) to[short, -*] (x3 |- nor.in 2)

		(Aandx1x2.out) -- (AandB.in 2)
		(Borx2x3.out) -- (AandB.in 1)
		(nor.out) -- (andAll.in 1)
		(AandB.out) -- (andAll.in 2)

		(andAll.out) node[right](y) {y}
		;
	\end{circuitikz}
\end{frame}
\section{Logischer Schluss}
\begin{frame}{Logische Folgerungen (Beweise)}
	\begin{itemize}
		\item logischer Schluss
		\begin{itemize}
			\item aus einer Reihe von Annahmen $A_1,\ldots,A_n$ (Voraussetzungen, Prämissen) soll eine Schlussfolgerung $B$ (Konklusion) gezogen werden
			\item formelmäßig ausgedrückt: $A_1 \land \ldots \land A_n \rightarrow B$
			\item Bezeichnung: $\left\{A_1, A_2, \ldots, A_n\right\} \models B$ bzw. $A_1, A_2, \ldots, A_n \models B$
		\end{itemize}
		\item Beweisfolge
		\begin{itemize}
			\item Aus einer Menge von Prämissen oder bereits abgeleiteten Aussagen werden durch Anwendung von Ableitungsregeln neue Aussagen abgeleitet, die im Weiteren ebenfalls benutzt werden können; entspricht Kettenschluss $((A_1 \rightarrow A_2) \land (A_2 \rightarrow A_3)) \rightarrow (A_1 \rightarrow A_3)$
			\item Ableitungsregeln sind entweder Äquivalenzregeln (äquivalente Ersetzungen gemäß Boolescher Rechenregeln) oder Schlussregeln (geeignete Tautologien)
		\end{itemize}
	\end{itemize}
\end{frame}

\begin{frame}{Schlussregeln}
	\begin{itemize}
		\item $(A \land (A \rightarrow B)) \rightarrow B$ (Modus ponens)\\
		"`wenn A gilt und aus A folgt B, dann gilt B"'
		\item $((A \rightarrow B) \land \neg B) \rightarrow \neg A$ (Modus tollens)\\
		"`wenn aus A B folgt und B nicht gilt, dann gilt A nicht"'
		\item $A \land B \rightarrow (A \land B)$ (Konjunktion)\\
		"`wenn A und B bereits gezeigt sind, dann gilt auch die Verknüpfung $A \land B$"'
		\item $A \land B \rightarrow A$ (Vereinfachung)\\
		"`wenn A und B gelten, dann gilt insbesondere auch A"'
		\item $A \rightarrow (A \lor B)$ (Ausdehnung)\\
		"`ist A bereits gezeigt, dann gilt auch A oder B"'
	\end{itemize}
\end{frame}

\begin{frame}{Bemerkungen und Beispiel}
	\begin{itemize}
		\item Zweckmäßige Tipps
		\begin{itemize}
			\item Formeln $\neg (A \land B)$ bzw. $\neg (A \lor B)$ umschreiben als $(\neg A \lor \neg B)$ bzw. $(\neg A \land \neg B)$
			\item $A \lor B$ überführen in Implikation $\neg A \rightarrow B$
			\item Ist als Konklusion B eine Implikation zu zeigen, d.h.\\
			$(A_1 \land \ldots \land A_k) \rightarrow (C \rightarrow D)$, dann kann stattdessen auch\\
			$(A_1 \land \ldots \land A_k) \rightarrow D$ gezeigt werden
		\end{itemize}
		\item Beispiel\\
			es gelten die Prämissen: $A, B \rightarrow C, (A \land B) \rightarrow (D \lor \neg C), B$;\\
			zeige: $D$
	\end{itemize}
\end{frame}

\begin{frame}[shrink=5]{Beweistypen (deduktiv)}
	\begin{columns}
		\column{.5\textwidth}
			\begin{enumerate}
				\item \underline{direkter Beweis} $A \rightarrow B$
				\item Beweis durch \underline{Umkehrschluss} (Kontraposition): zeige $\neg B \rightarrow \neg A$
				\item \underline{Widerspruchsbeweis:} zeige $(A \land \neg B) \rightarrow f$
				\item \underline{Äqzivalenzbeweis:}\\
				zu zeigen: $A \leftrightarrow B$\\
				stattdessen:\\
				$A \rightarrow B$ und $B \rightarrow A$ bzw. $A \rightarrow B$ und $\neg A \rightarrow \neg B$
				\item Beweis durch \underline{Fallunterscheidung:}\\
				zeige A durch $B \rightarrow A$ und $\neg B \rightarrow A$
				\item Beweis \underline{atomararer Aussagen:}\\
				zeige A durch $w \rightarrow A$ bzw. $\neg A \rightarrow f$
			\end{enumerate}
		\column{.5\textwidth}
			Erläuterungen
			\begin{enumerate}
				\item da $A \rightarrow B$ nur falsch sein kann für $I(A)=w$ und $I(B)=f$, genügt es, $A$ als wahr anzunehmen und $I(B)=w$ zu zeigen ($B$ abzuleiten)
				\item $(\neg B \rightarrow \neg A) \equiv A \rightarrow B$
				\item $(A \land \neg B) \rightarrow f \equiv A \rightarrow B$
				\item $A \leftrightarrow B \equiv (A \rightarrow B) \land (B \rightarrow A)$
				\item $A \equiv (B \rightarrow A) \land (\neg B \rightarrow A)$
				\item $A \equiv w \rightarrow A \equiv \neg A \rightarrow f$
			\end{enumerate}
	\end{columns}
\end{frame}

\begin{frame}{Exkurs: Methode der vollständigen Induktion}
	\begin{itemize}
		\item sei $A(n), n=0,1,\ldots$ eine Folge von Aussagen, die zu beweisen sind
		\item Induktionsbeweis
		\begin{itemize}
			\item[(1)] Induktionsbasis: zeige $A(0)$
			\item[(2)] zeige für alle $n: A(n) \rightarrow A(n+1)$
			\item alternative Formulierung
			\begin{itemize}
				\item[(2')] Induktionsvoraussetzung: $A(n)$ für ein beliebig gewähltes n
				\item[(3')] Induktionsschluss: zeige $A(n+1)$ aus $A(n)$
			\end{itemize}
		\end{itemize}
		\item Verallgemeinerte vollständige Induktion
		\begin{itemize}
			\item Zeige $A(0)$
			\item Zeige $A(0) \land A(1) \land \ldots \land A(n) \rightarrow A(n+1)$\\
			zum Beweis von $A(n)$ kann auf die Gültigkeit von $A(m)$ für beliebige $m<n$ zurückgegriffen werden
		\end{itemize}
	\end{itemize}
\end{frame}

\begin{frame}{Beispiele Beweistechniken}
	seien $a, b$ natürliche Zahlen
	\begin{itemize}
		\item direkter Beweis: wenn $a$ teilbar ist durch 6, dann ist $a$ auch teilbar durch 3
		\item Beweis durch Kontraposition: wenn $a^2$ ungerade, dann auch $a$
		\item Widerspruchsbeweis: wenn $a$ und $b$ gerade, dann auch $a \cdot b$
		\item Äquivalenzbeweis: $a$ ist gerade genau dann wenn $a^2$ gerade ist
		\item Beweis durch Fallunterscheidung: $a^2$ geteilt durch $4$ liefert Rest $1$ oder $0$
		\item Beweis atomarer Aussagen: $\sqrt{2}$ ist keine rationale Zahl
		\item Induktionsbeweis:
		\begin{itemize}
			\item zeige: $s_n := \sum_{i=1}^{n}i=\frac{n(n+1)}{2}$ (dazu auch direkten Beweis)
			\item Fibonacci-Folge: $f_1=1, f_2=1, f_n=f_{n-1}+f_{n-2} (n>2)$\\
			zeige: $f_n=\frac{1}{\sqrt{5}}(A^n-B^n)$ mit $A=\frac{1+\sqrt{5}}{2}$ und $B=\frac{1-\sqrt{5}}{2}$
		\end{itemize}
	\end{itemize}
\end{frame}
\section{Erfüllbarkeit aussagenlogischer Formeln}
\begin{frame}{Entscheidungsproblem SAT}
	\begin{itemize}
		\item Ist eine gegebene aussagenlogische Formel erfüllbar / nicht erfüllbar?
		\item Gesucht sind \emph{Algorithmen} (Verfahren, Handlungsanweisungen), die für eine (beliebige) aussagenlogische Formel als Eingabe nach endlich vielen Schritten mit (korrekter) Aussage ja/nein terminieren.
		\item trivialer Algorithmus: Berechnung der Wahrheitstafel $\rightarrow$ nicht effizient
		\item Bemerkung: jede aussagenlogische Formel lässt sich effizient in eine erfüllbarkeitsäquivalente Formel in KNF umschreiben (unter Einführung zusätzlicher Variablen) $\rightarrow$ betrachte im weiteren Formeln in KNF
	\end{itemize}
\end{frame}

\begin{frame}{Klauselmengen}
	\begin{itemize}
		\item Mengennotation für Formeln in KNF\\
		ersetze Klausel $L_{i1} \lor L_{i2} \lor \ldots$ durch $C_i = \left\{L_{i1}, L_{i2}, \ldots\right\}$ und Formel $F$ in KNF durch Klauselmenge $S=\left\{C_1, C_2, \ldots\right\}$
		\item Belegung $I$ lässt sich ebenfalls als Menge darstellen:\\
		z.B. $I(x_1)=w, I(x_2)=f, \ldots$\\
		dann Schreibweise $I=\left\{x_1, \overline{x_2}, \ldots\right\}$
		\item Belegung $I$ ist Modell der Klausel $C_i$ gdw. $I \cap C_i \neq \varnothing$.\\
		Belegung $I$ ist Modell der Klauselmenge $S$ gdw. $I \cap C_i \neq \varnothing$ für alle $C_i \in S$.
	\end{itemize}
\end{frame}

\begin{frame}{Hornformeln}
	\begin{itemize}
		\item Definition: $F$ ist Hornformel gdw. $F$ ist in KNF und jede Klausel enthält höchstens ein positives Literal.
		\item Beispiel: $F_1=(A \lor \neg B) \land (\neg C \lor \neg A \lor D) \land (\neg A \lor \neg B) \land D \land \neg E$
		\item Es gilt: jede Hornformel lässt sich äquivalent in eine Konjunktion von Implikationen ("`Regeln"') umformen
		\item Beispiel: $F_1=(B \rightarrow A) \land (A \land C \rightarrow D) \land (A \land B \rightarrow f) \land (w \rightarrow D) \land (E \rightarrow f)$
		\item Eine Menge von Hornklauseln heißt auch Logikprogramm
		\begin{itemize}
			\item Tatsachenklausel: ein positives und kein negatives Literal
			\item Prozedurklausel (Regel): ein positives und mindestens ein negatives Literal
			\item Zielklausel (Frageklausel): negative Klausel (ohne positives Literal)
		\end{itemize}
	\end{itemize}
\end{frame}

\begin{frame}{Beweismethodik}
	\begin{itemize}
		\item Zeige Gültigkeit einer regelbasierten aussagenlogischen Formel durch Nichterfüllbarkeit von $\neg F$
		\item Beispiel: es gelten die Prämissen $B, B \rightarrow A$ und $(A \land B) \rightarrow C$; logisches Schließen liefert $A$ und daraus auch $C$; also ist\\
		$F=(B \land (B \rightarrow A) \land (A \land B \rightarrow C)) \rightarrow (A \land C)$\\
		eine Tautologie. Es gilt\\
		$\neg F = B \land (B \rightarrow A) \land (A \land B \rightarrow C) \land (\neg A \lor \neg C)$\\
		das ist Hornformel, mit Zielklausel $(\neg A \lor \neg C) (\equiv (A \land C) \rightarrow f)$
		\item Anstelle die Gültigkeit von F direkt zu beweisen, zeige die Nichterfüllbarkeit der Hornformel $\neg F$
	\end{itemize}
\end{frame}

\begin{frame}{Erfüllbarkeitstest (Markierungsalgorithmus)}
	Sei F Konjunktion von Implikationen
	\begin{enumerate}
		\item Für alle Teilformeln der Art $w \rightarrow A$: markiere in allen Teilformeln auftretende $A$
		\item while (es gibt Teilformeln der Art
		\begin{enumerate}
			\item[(i)] $A_1 \land A_2 \land \ldots \land A_n \rightarrow B$ oder
			\item[(ii)] $A_1 \land A_2 \land \ldots \land A_n \rightarrow f$\\
			mit: alle $A_i$ markiert, $B$ nicht markiert)\\
			if (Fall (i)) markiere alle B\\
			else (Fall (ii)) Rückgabe "`unerfüllbar, STOP\\
			end
		\end{enumerate}
		end\\
		Rückgabe "`erfüllbar"'\\
		STOP
	\end{enumerate}
\end{frame}

\begin{frame}{Erfüllbarkeitstest (Hornklauselalgorithmus)}
	Sei $S$ Menge von Hornklauseln\\
	while ($S$ enthält eine positive Klausel $\left\{A_i\right\}$)\\
	\qquad entferne alle Klauseln, die $A_i$ enhalten\\
	\qquad und entferne $\overline{A_i}$ aus den verbliebenen Klauseln\\
	end\\
	if ($S$ enhält die leere Klausel $\Box$)\\
	\qquad Rückgabe "`unerfüllbar"'\\
	else Rückgabe "`erfüllbar"'\\
	end
\end{frame}

\begin{frame}{Erfüllbarkeit allgemeiner Klauselmengen: Davis-Putnam-Regeln}
	\begin{itemize}
		\item Sei $S=\left\{c_1, \ldots, c_k\right\}$ Klauselmenge, $L$ ein Literal, $\overline{L}$ das negierte Literal. Definiere:
		\begin{itemize}
			\item $S_L^+=\left\{c_j \in S \mid L \in c_j\right\}, S_L^-=\left\{c_j \in S \mid \overline{L} \in c_j\right\}, S_L^0=\left\{c_j \in S \mid L, \overline{L} \notin c_j\right\}$
			\item $POS_L(S)=S_L^0 \bigcup \left\{c_j \setminus \left\{L\right\} \mid c_j \in S_L^+\right\}; NEG_L(S)=S_L^0 \bigcup \left\{c_j \setminus \left\{\overline{L}\right\} \mid c_j \in S_L^-\right\}$
		\end{itemize}
		\item Es gilt: Die Klauselmenge $S$ ist erfüllbar gdw. wenigstens eine der beiden Klauselmengen $POS_L(S)$ und $NEG_L(S)$ erfüllbar ist
		\begin{itemize}
			\item Vorteil: Anzahl der Literale wird schrittweise reduziert
			\item Nachteil: in jedem Schritt verdoppelt sich i.A. die Zahl der Klauselmengen (vielfach aber auch nicht, s.u.)
		\end{itemize}
		\item Spezialfälle
		\begin{itemize}
			\item $S_L^-=\varnothing$: S erfüllbar gdw. $NEG_L(S)=S_L^0$ erfüllbar (analog für $S_L^+=\varnothing$)
			\item Unit-Regel: falls $\left\{L\right\} \in S$, dann $S$ erfüllbar gdw. $NEG_L(S)$ erfüllbar
		\end{itemize}
	\end{itemize}
\end{frame}

\begin{frame}{Resolution}
	\begin{itemize}
		\item Beweis der Unerfüllbarkeit allgemeiner Klauselmengen durch Herleitung der leeren Klausel
		\item Definition: Resolvente
		\begin{itemize}
			\item sei $L$ ein Literal und $\overline{L}$ das negierte Literal
			\item sei $c_1$ eine Klausel, die $L$ enthält, $c_2$ eine Klausel, die $\overline{L}$ enthält
			\item dann heißt die Klausel $res_L(c_1, c_2)=\left(c_1 \setminus \left\{L\right\}\right) \cup \left(c_2 \setminus \left\{\overline{L}\right\}\right)$
			\item Gesprochen: "`Resolvente der Klauseln $c_1$ und $c_2$ nach $L$"'
		\end{itemize}
		\item Resolutionslemma\\
		sei $I$ ein Modell von $c_1$ und $c_2$, dann ist $I$ auch Modell von $res_L(c_1, c_2)$, d.h. $I \cap c_1 \neq \varnothing \land I \cap c_2 \neq \varnothing \rightarrow I \cap res_L(c_1, c_2) \neq \varnothing$
	\end{itemize}
\end{frame}

\begin{frame}{Grundresolutionstheorem}
	\begin{itemize}
		\item Zu $S$ erfüllbarkeitsäquivalente Klauselmenge $RES_L(S)$
		\begin{itemize}
			\item sei $S$ Klauselmenge; $k$ Anzahl der Variablen; definiere:\\
			$RES_L(S)=S_L^0 \bigcup \left\{res_L(c_1, c_2) \mid c_1 \in S_L^+, c_2 \in S_L^-\right\}$
			\item beachte: $RES_L(S)$ enhält $L$ bzw. $\overline{L}$ nicht mehr
			\item es gilt: $S$ ist erfüllbar gdw. $RES_L(S)$ ist erfüllbar
		\end{itemize}
		\item Grundresolutionstheorem: eine Klauselmenge $S$ ist unerfüllbar gdw. $S$ lässt sich mittels Resolution widerlegen, d.h. die leere Klausel ist herleitbar
		\item Systematische Durchführung:\\
		$S_0=S$, $S_i=RES_{L_i}\left(S_{i-1}\right)$; alle $S_i, i \leq k$ sind erfüllbarkeitsäquivalent\\
		$S_k$ enthält kein Literal mehr, d.h. $S_k=\varnothing$ (erfüllbar) oder $S_k=\left\{\Box\right\}$ (unerfüllbar)
	\end{itemize}
\end{frame}
\section{Einführung: Sprachen und Grammatiken}

\begin{frame}{Sprachen: Bezeichnungen, Regeln}
	\begin{itemize}
		\item Alphabet $\Sigma$: Menge von Symbolen (Buchstaben)
		\item $\Sigma^*$:
		\begin{itemize}
			\item Menge aller Worte, die sich durch Hintereinanderschreiben ("`Konkatenation"') von Buchstaben aus $\Sigma$ bilden lassen
			\item Beispiel: $\Sigma=\left\{a,b\right\}, \Sigma^*=\left\{\varepsilon, a, b, aa, \ldots\right\}, \varepsilon$: "`leeres Wort"', $\varepsilon a=a \varepsilon = a$
			\item Länge eines Wortes $|w|: |a|=1, |\varepsilon|=0, |w_1w_2|=|w_1|+|w_2|$
		\end{itemize}
		\item Sprache $L$: Teilmenge von $\Sigma^*$
		\begin{itemize}
			\item Es gelten die Mengen-Verknüpfungen $L_1 \cup L_2, L_1 \cap L_2, L_1 \setminus L_2$
			\item Dazu die Konkatenation $L_1L_2=\left\{xy \mid x \in L_1 \land y \in L_2\right\}$
			\item Regeln:\\
			$L^0=\left\{\varepsilon\right\}, L^1=L, L^{n+1}=LL^n, L^*=\bigcup_{n\geq0}L^n, L^+=\bigcup_{n\geq1}L^n$
		\end{itemize}
	\end{itemize}
\end{frame}

\begin{frame}{Grammatiken}
	\begin{itemize}
		\item generierende Grammatik $G$: Sprache durch Regeln erzeugen
		\item Definition: $G=(V, \Sigma, P, S)$ mit\\
		$V$: endliche Menge von Variablen\\
		$\Sigma$: Terminalalphabet, $V \cap \Sigma = \varnothing$\\
		$P$: Menge von Regeln (Produktionen) der Art $u_1 \rightarrow u_2$, mit $u_1, u_2 \in \left(V \cup \Sigma\right)^*$\\
		$S$: Startvariable, $S \in V$
		\item Ableitung $S \underset{G}{\Rightarrow}^* w$
		\begin{itemize}
			\item Ableitungsschritt $u \underset{G}{\Rightarrow} v: u=xyz, v=xy'z, y\rightarrow y' \in P$
			\item $\underset{G}{\Rightarrow}^*$: reflexive und transitive Hülle der Relation $\underset{G}{\Rightarrow}$ (endliche Folge)
		\end{itemize}
		\item durch $G$ erzeugte Sprache: $L(G)=\left\{w \in \Sigma^* \mid S \underset{G}{\Rightarrow}^* w\right\}$
	\end{itemize}
\end{frame}

\begin{frame}{Chomsky-Hierarchie}
	\begin{itemize}
		\item Hierarchie von Grammatiken
		\begin{itemize}
			\item[\underline{Typ 0}:] Startwort besteht aus genau einer Variablen $S$,\\
			es gibt keine Regeln der Art $\varepsilon \rightarrow u$ abgeleitet werden; $S$ darf dann auf keiner rechten Seite einer Regel vorkommen
			\item[\underline{Typ 1:}] (kontextsensitiv) Typ 0, und für alle Regeln $w_1 \rightarrow w_2$ gilt: $|w_1| \leq |w_2|$;\\
			$S \rightarrow \varepsilon$ als Sonderregel erlaubt
			\item[\underline{Typ 2:}] (kontextfrei) Typ 1, und für alle Regeln $w_1 \rightarrow w_2$ gilt: $w_1$ ist eine einzelne Variable
			\item[\underline{Typ 3:}] (regulär) Typ 2, und für alle Regeln $w_1 \rightarrow w_2$ gilt: $w_2 \in \Sigma \cup \Sigma V$
		\end{itemize}
		\item \underline{$\varepsilon$-Sonderregel}: $\varepsilon$ darf nur aus Startsymbol $S$ abgeleitet werden; $S$ darf dann auf keiner rechten Seite einer Regel vorkommen
	\end{itemize}
\end{frame}

\begin{frame}{Beispiel Typ-1 Grammatik}
	\begin{align*}
		L&=\left\{a^nb^nc^n \mid n \geq 1\right\}\\
		G&=\left(\left\{S,B,C\right\},\left\{a,b,c\right\},P,S\right)\\
		P&=\{S \rightarrow aSBC \mid aBC,\\
		         &\qquad CB \rightarrow BC,\\
		         &\qquad aB \rightarrow ab,\\
		         &\qquad bB \rightarrow bb,\\
		         &\qquad bC \rightarrow bc,\\
		         &\qquad cC \rightarrow cc\}\\
	\end{align*}
\end{frame}

\begin{frame}{Ableitungsbaum}
	\begin{itemize}
		\item Darstellung der Ableitungsschritte als Baum
		\item Links- bzw. Rechtsableitungen
		\item Mehrdeutige Grammatiken
		\begin{itemize}
			\item zu ein und demselben Terminalwort existieren verschiedene (Links-) Ableitungen
			\item Eine Sprache $L$ heißt \emph{inhärent mehrdeutig}, falls keine eindeutige Grammatik für $L$ existiert
		\end{itemize}
		\item Berechnung von Ausdrücken:
		\begin{itemize}
			\item Operanden, die im Ableitungsbaum tiefer liegen, haben bei der Ausführung höhere Priorität
			\item Compiler erzeugt Ableitungsbaum bottom-up durch Ableitung des Startsymbols aus dem Terminalwort
			\item benutzt dazu \emph{reduktive} Grammatik: erzeugt aus der \emph{generativen} Grammatik durch Transponieren der Regeln
		\end{itemize}
	\end{itemize}
\end{frame}

\end{document}

