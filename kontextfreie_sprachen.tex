\section{Kontextfreie Sprachen}

\begin{frame}{Typ-2 Grammatiken: Normalformen}
	\begin{itemize}
		\item Vereinfachungen für kontextfreie Grammatiken $G$
		\begin{itemize}
			\item $G$ reduzieren: nutzlose Variablen eliminieren
			\item Kettenregeln $A \rightarrow B$ entfernen
		\end{itemize}
		\item Chomsky-Normalform
		\begin{itemize}
			\item zu jeder $\varepsilon$-freien, reduzierten kontextfreien Grammatik $G$ ohne Kettenregeln lässt sich eine äquivalente Grammatik $G'$ angeben, die nur Regeln der Art $A \rightarrow a$ bzw. $A \rightarrow BC$ enthält.
			\item Konstuktion: Ausgangspunkt $A\rightarrow X_1X_2 \ldots X_m, m\geq 2, X_i \in V \cup \Sigma$
			\begin{itemize}
				\item für $X_i=a\in \Sigma$ neue Variable $X_a$ mit $X_a \rightarrow a$ einführen
				\item verbleibende Regeln $A \rightarrow B_1B_2\ldots B_m, B_i \in V, m\geq 3$:\\
				ersetze $A \rightarrow B_1D_1, D_1\rightarrow B_2D_2, \ldots, D_{m-2}\rightarrow B_{m-1}B_m$ ($D_i$ neue Variablen)
			\end{itemize}
		\end{itemize}
		\item Greibach-Normalform: Regeln vom Typ $A \rightarrow aB_1B_2 \ldots B_m|a$
	\end{itemize}
\end{frame}

\begin{frame}{Beispiel: Erzeugung Chomsky-Normalform}
	$G=(\{S, A, B\}, \{a, b\}, P, S)$\\
	$P=\{$\\
	\qquad\qquad $S\rightarrow bA|aB$\\
	\qquad\qquad $A\rightarrow bAA|aS|a$\\
	\qquad\qquad $B\rightarrow aBB|bS|b$\\
	\qquad$\}$
\end{frame}

\note{
	\begin{tabular}{rl}
		alt & neu \\
		$S \rightarrow bA$ & $D_b \rightarrow b, S \rightarrow X_bA$ \\
		$S \rightarrow aB$ & $D_a \rightarrow a, S \rightarrow X_aB$ \\
		$A \rightarrow bAA$ & $A \rightarrow X_bAA$ (muss weiter ersetzt werden) \\
		$A \rightarrow aS$ & $A \rightarrow X_aS$ \\
		$B \rightarrow aBB$ & $B \rightarrow X_aBB$ (muss weiter ersetzt werden)\\
		$B \rightarrow bS$ & $B \rightarrow X_bS$ \\
		$A \rightarrow X_bAA$ & $A \rightarrow X_bD_1, D_1 \rightarrow AA$\\
		$B \rightarrow X_aBB$ & $B \rightarrow X_aD_2, D_2 \rightarrow BB$\\
	\end{tabular}
}

\begin{frame}{Worterkennung: CYK-Algorithmus}
	polynomiell auch für nichtdeterministische Sprachen
	\begin{itemize}
		\item Ziel: Worterkennung und Konstruktion Ableitungsbaum bottom-up
		\item Ausgangspunkt: $G$ in Chomsky-Normalform, $w=a_1a_2 \ldots a_n \in \Sigma^*$
		\item definiere $V[i,j]:=\{A \in V \mid A \underset{G}{\Rightarrow}^* w_{ij} \}$ mit\\
		$w_{ij}=a_ia_{i+1}\ldots a_{i+j-1}, |w_{ij}|=j, 1\leq i \leq n, 1\leq j \leq n+1-i$
		\item es gilt: $w\in L$ gdw. $S \in V[1,n]$
		\item schrittweise Berechnug (Rückführung auf kürzere Teilworte)\\
		$j=1: V[i,1]=\{X \in V \mid (X\rightarrow a_i) \in P\}$\\
		$j>1: V[i,j]=\bigcup_{l=1}^{j-1}\{A\in V\mid \exists (A\rightarrow BC): B \in V[i,l] \land C \in V[i+l, j-l] \}$
		\item Wortproblem mit polynomiellem Zeitaufwand $O(n^3)$ lösbar
		\item Grammatik ist mehrdeutig, falls ein $V[i,j]$ mehrere Einträge hat, die von $S$ aus erreicht werden können
	\end{itemize}
\end{frame}

\begin{frame}{Beispiel: CYK-Algorithmus}
	$L=\{a^nb^n\mid n\geq 1\}$\\
	$G=(\{A, B, C, S\}, \{a, b\}, P, S)$\\
	$P=\{S \rightarrow AB|AC, C\rightarrow SB, A\rightarrow a, B\rightarrow b\}$
\end{frame}

\note{
	Betrachten des Wortes $w=aabb$\\
	\begin{columns}
		\column{.5\textwidth}
		\begin{itemize}
			\item[$(a)$] $V[1,1]=\{A\}$
			\item[$(a)$] $V[2,1]=\{A\}$
			\item[$(b)$] $V[3,1]=\{B\}$
			\item[$(b)$] $V[4,1]=\{B\}$
			\item[$(aa)$] $V[1,2]=\varnothing$
		\end{itemize}
		\column{.5\textwidth}
		\begin{itemize}
			\item[$(ab)$] $V[2,2]=\{S\}$
			\item[$(bb)$] $V[3,2]=\varnothing$
			\item[$(aab)$] $V[1,3]=\varnothing$
			\item[$(abb)$] $V[2,3]=\{C\} (l=2)$
			\item[$(aabb)$] $V[1,4]=\{S\} (l=1)$
		\end{itemize}
	\end{columns}
}

\begin{frame}{Pumping-Lemma für kontextfreie Sprachen}
	\begin{itemize}
		\item Pumping-Lemma: für jede kontextfreie Sprache $L$ gilt\\
		es existiert eine natürliche Zahl $n$ so, dass für alle $z \in L$ mit $|z|\geq n$ gilt:\\
		$z$ lässt sich zerlegen in $z=uvwxy$ mit
		\begin{itemize}
			\item $|vx|\geq 1$
			\item $|vwx| \leq n$
			\item $uv^iwx^iy \in L, i=0, 1, 2, \ldots$
		\end{itemize}
		\item Beweis:
		\begin{itemize}
			\item wähle $n=2^m$, $m$: Anzahl der Nichtterminalsymbole, wenn sich G in Chomsky-Normalform befindet
			\item mindestens eine Variable $A$ muss sich im Ableitungsbaum nach spätestens $m$ Schritten wiederholen;\\
			$A\underset{G}{\Rightarrow}^*\omega_1 A \omega_2$, wähle $v: \omega_1 \underset{G}{\Rightarrow}^* v, x: \omega_2 \underset{G}{\Rightarrow}^* x, w: A \underset{G}{\Rightarrow}^* w$;\\
			diese Ableitungen enthalten jeweils keine Wiederholungen von Variablen
		\end{itemize}
	\end{itemize}
\end{frame}

\begin{frame}{Beispiel: Pumping-Lemma für kontextfreie Sprachen}
	$L=\{a^ib^ic^i \mid i\geq 1 \}$
	\begin{itemize}
		\item $L$ ist kontext-sensitiv: Beweis Typ-1-Grammatik\\
		$G=(V, \Sigma, P, S), V=\{S, B, C\}$\\
		$P=\{S\rightarrow aSBC | aBC, CB \rightarrow BC,$\\
		\qquad $aB \rightarrow ab, bB \rightarrow bb, bC \rightarrow bc, cC \rightarrow cc \}$
		\item $L$ ist nicht kontextfrei: Beweis Pumping-Lemma\\
		wäre $L$ kontextfrei, sei $n$ das "`Pumping-$n$"'\\
		wähle $z=a^nb^nc^n \in L, |z|=3n \geq n$\\
		es ex. eine Zerlegung $z=uvwxa$ mit $|vx|\geq 1, |vwx| \leq n$\\
		$\rightarrow$ $vwx$ kann höchstens zwei verschiedene Buchstaben enthalten\\
		$\rightarrow$ in $uv^iwx^iy$ werden ein oder zwei Buchstaben "`aufgepumpt"', der dritte nicht\\
		$\rightarrow$ $uv^iwx^iy \notin L$ für $i \neq 1$ \lightning 
	\end{itemize}
\end{frame}

\begin{frame}{Abschlusseigenschaften kontextfreier Sprachen}
	\begin{itemize}
		\item Kontextfreie Sprachen sind abgeschlossen unter
		\begin{itemize}
			\item Vereinigung: $L_1\cup L_2$\qquad $S \rightarrow S_1|S_2$
			\item Konkatenation: $L_1L_2$ \qquad $S\rightarrow S_1S_2$
			\item Stern-Produkt: $L^*$ \qquad $S_0\rightarrow S|\varepsilon, S\rightarrow S_1|S_1S$
			\item Spiegelung: ersetze in Chomsky-Normalform $A\rightarrow BC$ durch $A \rightarrow BC$
		\end{itemize}
		\item Kontextfreie Sprachen sind \underline{nicht} abgeschlossen unter
		\begin{itemize}
			\item [a)] Durchschnitt\\
			Gegenbeispiel: $L_1=\{a^kb^kc^l\mid k,l\geq 0\}, L_2=\{a^kb^lc^l\mid k,l\geq 0\}$ sind kontextfrei\\
			$L=L_1\cap L_2$ ist nicht kontextfrei
			\item [b)] Komplement\\
			$L_1 \cap L_2 = \overline{\overline{L_1} \cup \overline{L_2}}$: Abgeschlossenheit unter Vereinigung und Komplement würde zu Abgeschlossenheit unter Durchschnitt führen\\
			$\rightarrow$ Widerspruch zu a) \lightning
			
		\end{itemize}
	\end{itemize}
\end{frame}

\begin{frame}{Abschlusseigenschaften deterministisch kontextfreier Sprachen}
	\begin{enumerate}[a)]
		\item Deterministisch kontextfreie Sprachen sind abgeschlossen unter Komplementbildung (beachte: Akzeptanz durch Endzustand!)
		\item Deterministisch kontextfreie Sprachen sind \underline{nicht} abgeschlossen unter
		\begin{enumerate}[i.]
			\item Durchschnitt $L_1 \cap L_2$
			\item Konkatenation $L_1L_2$
			\item Stern-Produkt $L^*$
			\item Vereinigung $L_1 \cup L_2$
		\end{enumerate}
		\item $L_1 \cap L_2$ ist allerdings det. kontextfrei (bzw. kontextfrei) falls $L_1$ regulär und $L_2$ det. kontextfrei (bzw. kontextfrei) ist
	\end{enumerate}
\end{frame}