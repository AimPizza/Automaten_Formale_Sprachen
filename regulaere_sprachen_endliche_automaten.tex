\section{Reguläre Sprachen und Endliche Automaten}

% change settings of tikz for state machine drawings
\tikzset {
	->, % makes the edges directed
	>=stealth, % makes the arrow heads bold
	node distance=2cm and 2cm, % specifies the minimum distance between two nodes.
	every state/.style={thick, fill=gray!10}, %properties for every 'state' node
	initial text=$ $, % Remove text from start arrow
}

\begin{frame}{Endliche Automaten (Finite-State Machine (FSM), Finite Automaton (FA))}
	\begin{itemize}
		\item Grundkonzept
		\begin{itemize}
			\item FSM befinden sich jeweils in genau \emph{einem} aus einer \emph{endlichen Menge} von Zuständen
			\item FSM reagieren auf Eingaben, führen dabei gegebenenfalls Aktionen aus und wechseln ihren Zustand (Übergang, Transition)
		\end{itemize}
		\item Einsatz zur Spracherkennung
		\begin{itemize}
			\item Folge von Eingaben kann aufgefasst werden als Wort über einem Alphabet (Eingabe jeweils ein Buchstabe)
			\item Eingabenfolge entspricht einem Wort der vom FSM $A$ akzeptierten Sprache $L(A)$ gdw.
			\begin{itemize}
				\item das Wort vollständig gelesen wird und
				\item sich $A$ danach in einem der vorab definierten Endzustände befindet
			\end{itemize}
			\item Wort wird nicht generiert (Grammatik) sonder akzeptiert (als zugehörig erkannt)
		\end{itemize}
	\end{itemize}
\end{frame}

\begin{frame}{Deterministischer Endlicher Automat (DFA)}
	\begin{itemize}
		\item Allgemeine Spezifikation eines DFA $A$:\\
		$A=\left(S, \Sigma, \delta, s_0, F\right)$
		\begin{itemize}
			\item $S$: endliche Menge von Zuständen
			\item $\Sigma$: Alphabet
			\item $\delta:S \times \Sigma \rightarrow S$ Überführungsfunktion
			\item $s_0$: Startzustand, $s_0 \in S$
			\item $F$: Menge (akzeptierender) Endzustände, $F \subseteq S$
		\end{itemize}
		\item Graphische Darstellung: Zustände als Knoten eines Graphen (Zustandsgraph)\\
		\vspace{1em}
		\begin{tikzpicture}
			\node[state, initial] (s0) {$s_0$};
			\node[state, accepting, right=of s0] (sf) {$s_f$};
			\node[state, right=of sf] (s1) {$s_1$};
			\node[state, right=1cm of s1] (s2) {$s_2$};
			
			\draw	(s1) edge[above] node{a} (s2);
		\end{tikzpicture}
	\end{itemize}
\end{frame}

\begin{frame}{Beispiel DFA}
	\begin{align*}
		A &= (S, \Sigma, \delta, s_0, F)\\
		S &= \left\{s_0, s_1, s_2, s_3\right\}\\
		\Sigma &= \{a, b\}\\
		F &= \{s_3\}\\
		\delta &= \{(s_0, a, s_1), (s_0, b, s_3), (s_1, a, s_2), (s_1, b, s_0), \\
		& \qquad (s_2, a, s_3), (s_2, b, s_1), (s_3, a, s_0), (s_3, b, s_2)\}
	\end{align*}
\end{frame}

\note{
\begin{tikzpicture}
	\node[state, initial] (s0) {$s_0$};
	\node[state, right=of s0] (s1) {$s_1$};
	\node[state, below=of s1] (s2) {$s_2$};
	\node[state, accepting, below=of s0] (s3) {$s_3$};
	
	\draw (s0) edge[bend left, above] node{a} (s1);
	\draw (s0) edge[bend left, right] node{b} (s3);
	\draw (s1) edge[bend left, right] node{a} (s2);
	\draw (s1) edge[bend left, below] node{b} (s0);
	\draw (s2) edge[bend left, below] node{a} (s3);
	\draw (s2) edge[bend left, left] node{b} (s1);
	\draw (s3) edge[bend left, left] node{a} (s0);
	\draw (s3) edge[bend left, above] node{b} (s2);
	
\end{tikzpicture}
}

\begin{frame}{Durch DFA akzeptierte Sprachen: Reguläre Sprachen}
	\begin{itemize}
		\item Konfiguration des Automaten: $A: (s,u)$
		\begin{itemize}
			\item $s$: aktueller Zustand
			\item $u$: noch zu lesendes Teilwort
		\end{itemize}
		\item Konfigurationsübergang\\
		\quad $(s, av) \mapsto (s', v) \leftrightarrow \delta(s,a) = s'$
		\item Durch $A$ akzeptierte Sprache\\
		\quad $L(A) = \left\{w \in \Sigma^* \mid (s_0,w) \mapsto^* (s_f, \varepsilon), s_f \in F \right\}$
		\item es gilt: reguläre Sprachen sind Teilmenge der Typ-3-Sprachen\\
		\begin{itemize}
			\item[Beweis:] konstruiere zu $A$ einen korrespondierende Grammatik $G=(S,\Sigma,P,S_0)$\\
			\item[$S:$] $S_i=s_i$, insbesondere $S_0=s_0$
			\item[$P:$] für jeden Übergang $\delta(s,a)=s'$ die Regel $S \rightarrow aS'$\\
			zusätzlich die Regeln $S \rightarrow a$, falls $s'$ Endzustand und $S_0 \rightarrow \varepsilon$, falls $s_0$ Endzustand
		\end{itemize}
	\end{itemize}
\end{frame}

\begin{frame}{Nichtdeterministische endliche Automaten (NFA)}
	\begin{itemize}
		\item Ausgangspunkt: Typ-3-Grammatiken erlauben Regeln der Art\\
		\quad $S_1 \rightarrow aS_2 \mid aS_3$
		\item korrespondierende nichtdeterministische Automaten: zu
		einem Zustand $s_1$ kann es bei Eingabe $a$ Übergänge zu
		verschiedenen Zuständen $s_2, s_3$ geben
		\item $\delta$ ist dann keine Abbildung, sondern eine allgemeinere Transitionsrelation:\\
		\quad $\delta \subseteq (S \times \Sigma) \times S$
		\item akzeptierte Sprache:\\
		\quad $L(A)=\left\{w \in \Sigma^* \mid (s_0,w) \mapsto^* (S_f, \varepsilon), s_f \in F, s_0 \in S_0 \right\}$
		\begin{itemize}
			\item $w$ wird akzeptiert, falls es eine akzeptierende Folge von Konfigurationsübergängen gibt; andere Folgen müssen nicht akzeptierend sein
			\item $S_0:$ Menge von Startzuständen (mehrere sind erlaubt)
		\end{itemize}
	\end{itemize}
\end{frame}

\begin{frame}{Beispiel NFA}
	\label{NFA_Beispiel}
	\begin{tikzpicture}
		\node[state, initial] (s0) {$s_0$};
		\node[state, initial, initial where=above, right=of s0] (s1) {$s_1$};
		\node[state, accepting, right=of s1] (s2) {$s_2$};
		
		\draw (s0) edge[loop, above] node{$0|1$} (s0);
		\draw (s0) edge[above] node{$0$} (s1);
		\draw (s1) edge[above] node{$0$} (s2);
	\end{tikzpicture}\\
	$A=(S, \Sigma, \delta, S_0, F)$\\
	$S=\{s_0, s_1, s_2\}$ \\
	$S_0=\{s_0, s_1\}$\\
	$\Sigma=\{0,1\}$\\
	$F=\{s_2\}$ \\
	$\delta=\{(s_0,0,s_0), (s_0, 1, s_0), (s_0, 0, s_1), (s_1, 0, s_2)\}$\\ 
	Akzeptierte Sprache: $L(A)=\left\{w \in \Sigma^* \mid w=0 \lor w=x00, x\in \Sigma^*\right\}$
\end{frame}

\begin{frame}{Simulation des Verhaltens von NFAs}
	\begin{itemize}
		\item Backtracking: Tiefensuche
		\begin{itemize}
			\item Jeweis eine mögliche Alternative wird weiter verfolgt $\rightarrow$ Tiefensuchbaum
			\item bei "`Sackgasse"': zurück (aufwärts im Baum) bis zur nächsten freien Alternative
			\item Abbruch, wenn akzeptierende Konfigurationsfolge gefunden oder sonst alle möglichen Konfigurationsfolgen untersucht
		\end{itemize}
		\item Breitensuche:
		\begin{itemize}
			\item parallele schrittweise Verfolgung aller Verzweigungen (Breitensuchbaum)
			\item Darstellung von $\delta$ als mengenwertige Funktion $\delta: S \times \Sigma \rightarrow \mathcal{P}(S)$
			\item Erweiterung auf Folge von Konfigurationsübergängen\\
			\quad $\hat{\delta}: \mathcal{P}(S) \times \Sigma^* \mapsto \mathcal{P}(S)$\\
			\quad $\hat{\delta}(S', \varepsilon) = S'$ \quad $\hat{\delta}(S', av)=\hat{\delta}\left(\bigcup_{s \in S}, \delta(s,a),v\right)$ \quad $(S' \in \mathcal{P}(S))$
			\item Akzeptierte Sprache\\
			\quad $L(A)=\left\{w \in \Sigma^* \mid \hat{\delta}(S_0,w)\cap F \neq \varnothing\right\}$
		\end{itemize}
	\end{itemize}
\end{frame}

\begin{frame}{Äquivalenz von DFA und NFA}
	\begin{itemize}
		\item jede durch einen NFA akzeptierte Sprache kann auch durch einen DFA akzeptiert werden (umgekehrt sowieso)
		\item Konstruktiver Beweis: Potenzmengenkonstruktion für \underline{DFA} $A_d$ äquivalent zu NFA $(S, \Sigma, \delta, S_0, F)$\\
		$A_d=\left(S_d, \Sigma, \delta_d, s_0^d, F_d\right)$ mit\\
		\quad $S_d=\mathcal{P}(S)$; 
		\quad $s_0^d=S_0 \in \mathcal{P}(S)$;
		\quad $\delta_d(S', a)=\bigcup_{s \in S}$
		\quad $\delta(s, a) \in \mathcal{P}(S)$\\
		\quad $F_d = \left\{S' \in \mathcal{P}(S) \mid S' \cap F \neq \varnothing\right\} \subseteq \mathcal{P}(S)$
		\item es gilt: $w \in L(A)$ gdw. $\hat{\delta}(S_0, w) \cap F \neq \varnothing$\\
		\quad gdw. $(s_0^d, w) \mapsto^* (s_f^d, \varepsilon), s_f^d \in F_d$ gdw. $w \in L(A_d)$
		\item Nutzen: mit NFA kann eine Sprache leichter modelliert werden $\rightarrow$ zur Anwendung in DFA umrechnen
	\end{itemize}
\end{frame}

\begin{frame}{Algorithmus zur Umwandlung NFA in DFA (verbal)}
	NFA: $A_n=(S_n, \Sigma, \delta_n, S_0^n, F_n)$, DFA: $A_d=(S_d, \Sigma, \delta_d, s_0^d, F_d)$
	\begin{enumerate}
		\item $S_d$ ist die Potenzmenge von $S_n$
		\item $s_0^d$ ist das Element aus $S_d$, das die Kombination aller Startzustände aus $S_0^n$ repräsentiert. Aus (möglicherweise) vielen Startzuständen in $S_n$ wird somit genau ein Startzustand $s_0^d$.
		\item Jede Kombination in $S_d$, die mindestens einen Endzustand aus $F_n$ enthält, wird zum Endzustand in $F_d$.
		\item Für jede Zustandskombination in $S_d$ wird ermittelt, welche Zustände mit dem Eingabesymbol $a \in \Sigma$ erreichbar sind. Diese Kombination wird zur Übergangsfunktion $\delta_d$ hinzugefügt.
	\end{enumerate}
\end{frame}

\begin{frame}{Beispiel: Umwandlung NFA in DFA (Berechnung)}
	Beispiel zur Umwandlung des NFA aus Folie \ref{NFA_Beispiel} in DFA:
	\begin{itemize}
		\item $S_d=\{\varnothing, s_0, s_1, s_2, s_0s_1, s_0s_2, s_1s_2, s_0s_1s_2\}$
		\item $s_0^d=s_0s_1$
		\item $\Sigma=\{0,1\}$ (unverändert)
		\item $F_d=\{s_2, s_0s_2, s_1s_2, s_0s_1s_2\}$
		\item $\delta_d=\{(\varnothing, 0, \varnothing), (\varnothing, 1, \varnothing), (s_0, 0, s_0s_1), (s_0, 1, s_0), (s_1, 0, s_2), (s_1, 1, \varnothing),$\\
		\qquad $(s_2, 0, \varnothing), (s_2, 1, \varnothing), (s_0s_1, 0, s_0s_1s_2),(s_0s_1, 1, s_0),$\\
		\qquad $(s_0s_2, 0, s_0s_1), (s_0s_2, 1, s_0), (s_1s_2, 0, s_2), (s_1s_2, 1, \varnothing),$\\
		\qquad $(s_0s_1s_2, 0, s_0s_1s_2), (s_0s_1s_2, 1, s_0)\}$
		\item Für den DFA $A_d$ sind die Zustände $\varnothing, s_1, s_2$ und $s_1s_2$ nicht relevant, da sie vom Startzustand $s_0s_1$ aus nicht erreicht werden können.
	\end{itemize}
\end{frame}

\begin{frame}{Beispiel: Umwandlung NFA in DFA (Zustandsgraph)}
	\centering
	\tikzset {node distance=1cm}
	\begin{tikzpicture}
		\node[state] (s0) {$s_0$};
		
		\node[state, accepting, below=of s0] (s0s2) {$s_0s_2$};
		\node[state, initial, initial where=right, right=of s0s2] (s0s1) {$s_0s_1$};
		\node[state, accepting, right=of s0] (s0s1s2) {$s_0s_1s_2$};
		
		\node[state, below=of s0s2] (s1) {$s_1$};
		\node[state, accepting, right=of s1] (s2) {$s_2$};
		\node[state, accepting, below=of s2] (s1s2) {$s_1s_2$};
		
		\node[state, below=of s1] (leer) {$\varnothing$};
		
		\draw	(leer) edge[loop left] node{0,1} (leer)
				(s0) edge[bend left, below left] node{0} (s0s1)
				(s0) edge[loop left] node{1} (s0)
				(s1) edge[above] node{0} (s2)
				(s1) edge[right] node{1} (leer)
				(s2) edge[below right] node{0,1} (leer)
				(s0s1) edge[right] node{0} (s0s1s2)
				(s0s1) edge[bend left, above right] node{1} (s0)
				(s0s2) edge[below] node{0} (s0s1)
				(s0s2) edge[left] node{1} (s0)
				(s1s2) edge[right] node{0} (s2)
				(s1s2) edge[below] node{1} (leer)
				(s0s1s2) edge[loop right] node{0} (s0s1s2)
				(s0s1s2) edge[above] node{1} (s0);
	\end{tikzpicture}
\end{frame}

\begin{frame}{Äquivalenz von Typ-3-Sprachen und regulären Sprachen}
	\begin{itemize}
		\item Menge der regulären Sprachen Teilmenge der Typ-3-Sprachen:\\
		\quad $\rightarrow$ schon gezeigt (DFA)
		\item Menge der Typ-3-Sprachen Teilmenge der regulären Sprachen:\\
		Beweis: konstruiere zu gegebener Typ-3-Grammatik einen (i.A. nichtdeterministischen) FA\\
		\begin{tabbing}
			\hspace{5cm}\=\kill
		$G=(V,\Sigma, P, S_0)$	\> $FA=(S, \Sigma, \delta, s_0, F)$\\ 
			\> intialisiere: $S=V \cup \left\{f\right\}, s_0=S_0, \delta=\varnothing$\\ 
			\> $F=\left\{f\right\}$ (neues Symbol $f$)\\ 
		Regel $A \rightarrow aB$	\> füge ein: $(A, a, B) \in \delta$\\ 
		Regel $A \rightarrow a$	\> füge ein: $(A, a, f) \in \delta$\\ 
		Regel $S_0 \rightarrow \varepsilon \mid S_1$	\> füge $S_0$ in $F$ ein; für $S_1$ die Regeln für $S_1$\\ 
			\> einsetzen (Kettenregel); $S_1$ verbleibt als Zustand
		\end{tabbing} 
	\end{itemize}
\end{frame}

\begin{frame}{Minimalautomat}
	\begin{itemize}
		\item Ausgangspunkt: ein DFA;\\
		Ziel: konstruiere einen äquivalenten DFA mit geringster Anzahl von Zuständen
		\item Konstruktion
		\begin{enumerate}
			\item Beginne mit einer Partition $P_1=\left\{S_{11}, S_{12}\right\}$ der Zustandsmenge $S$ in $S_{11}=F, S_{12}=S \setminus F$
			\item bilde aus $P_i=\left\{S_{i1},\ldots,S_{ik}\right\}$ die Partition $P_{i+1}$ durch folgende Verfeinerung:\\
			teile $S_{ij}$, falls es in $S_{ij}$ Zustände $s$ und $s'$ gibt mit:\\
			für ein $a \in \Sigma$ liegen $\delta(s, a)$ und $\delta(s', a)$ in unterschiedlichen Blöcken $S_{il}$ von $P_i$
			\item Minimalautomat erreicht, wenn $P_{i+1}=P_i$;\\
			die $S_{ij}$ bilden dann die Zustände des Minimalautomaten
		\end{enumerate}
		\item Minimalautomat kann alternativ auch durch schrittweises Zusammenfassen von Zuständen konstruiert werden (insbesondere falls Menge der Nichtendzustände leer)
	\end{itemize}
\end{frame}

\begin{frame}{Beispiel: Minimalautomat (Ausgangssituation)}
	\centering
	\begin{tikzpicture}
		\node[state, initial] (s0) {$s_0$};
		\node[state, right=of s0] (s2) {$s_2$};
		\node[state, above=of s2] (s1) {$s_1$};
		\node[state, below=of s2] (s3) {$s_3$};
		\node[state, accepting, right=of s2] (s4) {$s_4$};
		
		\draw	(s0) edge[above left] node{0} (s1)
				(s0) edge[above] node{1} (s2)
				(s1) edge[above right] node{0} (s4)
				(s1) edge[right] node{1} (s2)
				(s2) edge[right] node{0} (s3)
				(s2) edge[loop right] node{1} (s2)
				(s3) edge[below right] node{0} (s4)
				(s3) edge[below left] node{1} (s0)
				(s4) edge[loop right] node{0,1} (s4);
	\end{tikzpicture}
\end{frame}

\note{
	\scriptsize
	\begin{itemize}
		\item Tabelle aufstellen: Zustände paarweise
		\item Diagonale: nicht markierbar (identischer Zustand)
		\item obere Hälfte: nicht markierbar (Duplikat von unterer Hälfte)
		\item Paare von Endzuständen und Nicht-Endzuständen markieren
		\item Für jedes verbliebene Paar: Prüfen, ob mindestens ein Eingabezeichen in ein Zustandspaar führt, das bereits markiert ist. Wenn ja, dann auch dieses Paar markieren
		\item Paare ohne Markierung können zusammengefasst werden
		\item Bei großen Automaten $\rightarrow$ iterativ, bis keine Änderung mehr auftritt
	\end{itemize}
}

\begin{frame}{Beispiel: Minimalautomat (Ergebnis)}
	\begin{columns}
		\column{.5\textwidth}
		\centering
		\begin{tabular}{|r||c|c|c|c|c|}
			\hline
			$s_0$ & - & - & - & - & - \\
			\hline
			$s_1$ & X & - & - & - & - \\
			\hline
			$s_2$ &  & X & - & - & - \\
			\hline
			$s_3$ & X &  & X & - & - \\
			\hline
			$s_4$ & X & X & X & X & - \\
			\hline
			\hline
			& $s_0$ & $s_1$ & $s_2$ & $s_3$ & $s_4$ \\
			\hline
		\end{tabular}
		\column{.5\textwidth}
		\centering
		\begin{tikzpicture}
			\node[state, initial] (s0s2) {$s_0s_2$};
			\node[state, below=of s0s2] (s1s3) {$s_1s_3$};
			\node[state, accepting, below=of s1s3] (s4) {$s_4$};
			
			\draw	(s0s2) edge[bend left, right] node{0} (s1s3)
					(s0s2) edge[loop right] node{1} (s0s2)
					(s1s3) edge[right] node{0} (s4)
					(s1s3) edge[bend left, left] node{1} (s0s2)
					(s4) edge[loop right] node{0,1} (s4);
		\end{tikzpicture}
	\end{columns}
\end{frame}

\begin{frame}{Abgeschlossenheit regulärer Sprachen}
	Seien $L, L_1, L_2$ reguläre Sprachen. Dann sind auch die folgenden Sprachen regulär:
	\begin{itemize}
		\item jede endliche Sprache $L \subset \Sigma^*$
		\item $L = \Sigma^*$
		\item $L_1 \cup L_2$
		\item $L_1 \setminus L_2$ (und damit auch Komplement $\Sigma^* \setminus L$)
		\item $L_1 \cap L_2$
		\item $L_1L_2$
		\item $L^*$
		\item $\tilde{L}$ (gespiegelte Sprache)
	\end{itemize}
	\emph{Kleene's Theorem}\\
		Eine Sprache $L$ ist regulär gdw. sie lässt sich durch endlich viele
		Anwendungen der Operationen Vereinigung, Konkatenation und $*$ aus
		einer endlichen Sprache erzeugen.
\end{frame}

\begin{frame}{Reguläre Ausdrücke}
	\begin{itemize}
		\item Zweck: Mittel zur Beschreibung regulärer Sprachen
		\item Induktive Definition regulärer Ausdrücke $\alpha$ über $\Sigma$
		\begin{itemize}
			\item $\varnothing $ ist ein regulärer Ausdruck: $L(\varnothing)=\varnothing$
			\item $\varepsilon$ ist ein regulärer Ausdruck: $L(\varepsilon)=\left\{\varepsilon\right\}$
			\item für jedes $a \in \Sigma$ ist $a$ ein regulärer Ausdruck: $L(a)=\left\{a\right\}$
			\item sind $\alpha$ und $\beta$ reguläre Ausdrücke, dann auch $\alpha^*$, $\alpha\beta$ und $\alpha | \beta$\\
			\quad $L(\alpha^*)=(L(\alpha))^*,$ $L(\alpha|\beta)=L(\alpha)\cup L(\beta),$ $L(\alpha\beta)=L(\alpha)L(\beta)$
			\item nur die so gebildeten Ausdrücke sind regulär
		\end{itemize}
		\item "`Rechenregeln"' bzw. Bezeichnungen\\
		\quad $\varnothing|\alpha=\alpha;$ $\varnothing\alpha=\varnothing;$ $\varnothing^*=\varepsilon;$ $\varepsilon\alpha=\alpha;$ $\alpha|\alpha=\alpha;$ $\alpha^+:=\alpha\alpha^*;$ $\alpha^+|\varepsilon=\alpha^*$
		\item Kleene's Theorem (umformuliert)\\
		\quad Eine Sprache $L'$ ist regulär gdw.\\
		\quad es existiert ein regulärer Ausdruck $\alpha$ mit $L'=L(\alpha)$.
	\end{itemize}
\end{frame}

\begin{frame}{$\varepsilon$-Automaten ($\varepsilon$FA)}
	\begin{itemize}
		\item Definition: Erweiterung von NFA; es werden Zustandsübergänge zugelassen ohne Lesen eines Zeichens\\
		\quad $\delta\subseteq S \times \left(\Sigma \cup \left\{\varepsilon\right\}\right) \times S$
		\item Nutzen:
		\begin{itemize}
			\item besonders geeignet zum Zusammensetzen von Automaten aus Teilautomaten
			\item Umsetzung regulärer Ausdrücke in $\varepsilon$-Automaten, danach in NFA und damit auch in DFA umrechenbar
		\end{itemize}
		\item Beispiel: $\alpha=a^*b^*$
	\end{itemize}
\end{frame}

\begin{frame}{Zusammengesetzte Automaten}
	
	\begin{columns}
		\begin{column}{0.5\textwidth}  %%<--- here
			\begin{tikzpicture}
				\node[state, initial] (s0) {$s_0$};
				\node[state, right=1.5cm of s0, accepting] (sf) {$s_f$};
				
				\draw	(s0) edge[above] node{$\alpha$} (sf);
			\end{tikzpicture}
		\end{column}
		\begin{column}{0.5\textwidth}
			Top-Down-Zugang: zerlege den durch den regulären Ausdruck $\alpha$ beschriebenen Automaten in Teilbestandteile
		\end{column}
	\end{columns}
	
	\begin{tikzpicture}
		\node[state] (s11l) {$s_1$};
		\node[state, right=1.5cm of s11l] (s21l) {$s_2$};
		\draw (s11l) edge[above] node{$\alpha_1|\alpha_2$} (s21l);
		
		\coordinate[right=1cm of s21l] (a1l);
		\coordinate[right=1cm of a1l] (a1r);
		\draw [-Triangle, line width=3pt](a1l) -- (a1r);
		
		\node[state, right=1cm of a1r] (s11r) {$s_1$};
		\node[state, right=1.5cm of s11r] (s21r) {$s_2$};
		\draw (s11r) edge[above, bend left] node{$\alpha_1$} (s21r);
		\draw (s11r) edge[below, bend right] node{$\alpha_2$} (s21r);
		
		
		\node[state, below=0.75cm of s11l] (s12l) {$s_1$};
		\node[state, right=1.5cm of s12l] (s22l) {$s_2$};
		\draw (s12l) edge[above] node{$\alpha_1\alpha_2$} (s22l);
		
		\coordinate[right=1cm of s22l] (a2l);
		\coordinate[right=1cm of a2l] (a2r);
		\draw [-Triangle, line width=3pt](a2l) -- (a2r);
		
		\node[state, right=1cm of a2r] (s12r) {$s_1$};
		\node[state, right=1cm of s12r] (sq2r) {$q$};
		\node[state, right=1cm of sq2r] (s22r) {$s_2$};
		\draw (s12r) edge[above] node{$\alpha_1$} (sq2r);
		\draw (sq2r) edge[above] node{$\alpha_2$} (s22r);
		
		\node[state, below=0.5cm of s12l] (s13l) {$s_1$};
		\node[state, right=1.5cm of s13l] (s23l) {$s_2$};
		\draw (s13l) edge[above] node{$\alpha^*$} (s23l);
		
		\coordinate[right=1cm of s23l] (a3l);
		\coordinate[right=1cm of a3l] (a3r);
		\draw [-Triangle, line width=3pt](a3l) -- (a3r);
		
		\node[state, right=1cm of a3r] (s13r) {$s_1$};
		\node[state, right=1cm of s13r] (sq13r) {$q_1$};
		\node[state, right=1cm of sq13r] (sq23r) {$q_2$};
		\node[state, right=1cm of sq23r] (s23r) {$s_2$};
		\draw (s13r) edge[above] node{$\varepsilon$} (sq13r);
		\draw (sq13r) edge[above, bend left] node{$\varepsilon$} (sq23r);
		\draw (sq23r) edge[below, bend left] node{$\alpha$} (sq13r);
		\draw (sq23r) edge[above] node{$\varepsilon$} (s23r);
		
	\end{tikzpicture}
\end{frame}

\begin{frame}{Äquivalenz von $\varepsilon$FA und NFA}
	Konstruktion eines zu einem $\varepsilon$FA äquivalenten NFA
	\begin{enumerate}
		\item erweitere $A$ um zwei Zustände $i$ und $f$; $\varepsilon$-Übergänge von $i$ zu allen Anfangszuständen aus $S_0$, sowie von allen Endzuständen aus $F$ zu $f$
		\item eliminiere alle $\varepsilon$-Zyklen; die betroffenen Zustände werden zu einem neuen Zustand zusammengefasst; verbleibende $\varepsilon$-Übergänge des neuen Zustands auf sich selbst werden entfernt
		\item Neue Übergänge einfügen: gibt es einen $\varepsilon$-Übergang von $s_1$ nach $s_2$, und einen $a$-Übergang von $s_2$ nach $s_3$ (bzw. auch $a$-Übergang von $s_1$ nach $s_2$, und einen $\varepsilon$-Übergang von $s_2$ nach $s_3$), dann füge einen Übergang $(s_1,a,s_3)$ ein; iterieren über neu hinzugefügte Übergänge
		\item Endzustandsmenge bestimmen: Endzustände sind alle Zustände von
		denen aus $f$ über eine Folge von $\varepsilon$-Übergängen erreichbar ist
		\item alle $\varepsilon$-Übergänge eliminieren
	\end{enumerate}
\end{frame}

\begin{frame}{Pumping-Lemma für reguläre Sprachen}
	\begin{itemize}
		\item Wie kann man feststellen dass eine Sprache \underline{nicht} regulär ist?
		\item Für reguläre Sprachen $L$ gilt das Pumping-Lemma:\\
		Sei L regulär. Dann gibt es eine natürliche Zahl n so dass $\forall x \in L \textrm{ mit } |x|\geq n$ gilt:\\
		$x$ lässt sich so in Teilworte $u, v, w$ zerlegen, d.h. $x=uvw$, dass
		gilt:
		\begin{enumerate} [i.]
			\item $|v| \geq 1$
			\item $|uv| \leq n$
			\item $uv^iw \in L\quad \forall i = 0, 1, 2, ...$ ("`aufpumpen"' von $x$)
		\end{enumerate}
		\item Genügt $L$ diesem Lemma nicht, dann kann $L$ nicht regulär sein
	\end{itemize}
	Beispiel: $L=\{a^nb^n\mid n\geq 1\}$
\end{frame}

\begin{frame}{Entscheidungsprobleme für reguläre Sprachen}
	Seien $G$ bzw. $G_1$ und $G_2$ Typ-3 Grammatiken. Die folgenden
	Probleme sind algorithmisch entscheidbar:
	\begin{itemize}
		\item Wortproblem $w \in L(G)?$
		\item Leerheitsproblem $L(G) = \varnothing?$
		\item Endlichkeitsproblem $|L(G)|=\infty ?$
		\item Schnittproblem $L(G_1) \cap L(G_2) = \varnothing ?$
		\item Äquivalenzproblem $L(G_1) = L(G_2)?$
	\end{itemize}
\end{frame}

\begin{frame}{Zusamenfassung}
	\begin{itemize}
		\item reguläre Grammatik ($w_1 \rightarrow w_2 \in P$)
		\begin{itemize}
			\item $|w_1|\leq|w_2|$ (Wort wird nicht kürzer, Typ-1)
			\item $w_1 \in V$ ($w_1$ ist einzelne Variable, Typ-2)
			\item $w_2 \in \Sigma \cup \Sigma V$ ($A \rightarrow a$ oder $A \rightarrow aB$, Typ-3)
		\end{itemize}
		\item DFA
		\item NFA
		\item $\varepsilon$FA
		\item reguläre Ausdrücke
	\end{itemize}
\end{frame}