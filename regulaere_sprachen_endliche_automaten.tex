\section{Reguläre Sprachen und Endliche Automaten}

% change settings of tikz for state machine drawings
\tikzset {
	->, % makes the edges directed
	>=stealth, % makes the arrow heads bold
	node distance=2cm and 2cm, % specifies the minimum distance between two nodes.
	every state/.style={thick, fill=gray!10}, %properties for every 'state' node
	initial text=$ $, % Remove text from start arrow
}

\begin{frame}{Endliche Automaten (Finite-State Machine (FSM), Finite Automaton (FA))}
	\begin{itemize}
		\item Grundkonzept
		\begin{itemize}
			\item FSM befinden sich jeweils in genau einem aus einer endlichen Menge von Zuständen
			\item FSM reagieren auf Eingaben, führen dabei gegebenenfalls Aktionen aus und wechseln ihren Zustand (Transition)
		\end{itemize}
		\item Einsatz zur Spracherkennung
		\begin{itemize}
			\item Folge von Eingaben kann aufgefasst werden als Wort über einem Alphabet (Eingabe jeweils ein Buchstabe)
			\item Eingabenfolge entspricht einem Wort der vom FSM $A$ akzeptierten Sprache $L(A)$ gdw.
			\begin{itemize}
				\item das Wort vollständig gelesen wird und
				\item sich $A$ danach in einem der vorab definierten Endzustände befindet
			\end{itemize}
			\item Wort wird nicht generiert (Grammatik) sonder akzeptiert (als zugehörig erkannt)
		\end{itemize}
	\end{itemize}
\end{frame}

\begin{frame}{Deterministischer Endlicher Automat (DFA)}
	\begin{itemize}
		\item Allgemeine Spezifikation eines DFA $A$:\\
		$A=\left(S, \Sigma, \delta, s_0, F\right)$
		\begin{itemize}
			\item $S$: endliche Menge von Zuständen
			\item $\Sigma$: Alphabet
			\item $\delta:S \times \Sigma \rightarrow S$ Überführungsfunktion
			\item $s_0$: Startzustand, $s_0 \in S$
			\item $F$: Menge (akzeptierender) Endzustände, $F \subseteq S$
		\end{itemize}
		\item Graphische Darstellung: Zustände als Knoten eines Graphen\\
		\vspace{1em}
		\begin{tikzpicture}
			\node[state, initial] (s0) {$s_0$};
			\node[state, accepting, right of=s0] (sf) {$s_f$};
			\node[state, right=of sf] (s1) {$s_1$};
			\node[state, right=of s1] (s2) {$s_2$};
			
			\draw	(s1) edge[above] node{a} (s2);
		\end{tikzpicture}
	\end{itemize}
\end{frame}