\section{Aussagenlogik}
\begin{frame}{Aussagenlogik}
	Gliederung
	\begin{itemize}
		\item Aussagenlogische Formeln
		\item Mengen/Relationen
		\item Boolesche Rechenregeln
		\item Beweistechniken
		\item Erfüllbarkeit aussagenlogischer Formeln
	\end{itemize}
\end{frame}

\begin{frame}{Aussagenlogik: Syntax}
	\begin{itemize}
		\item \emph{Aussage:} Satz, der entweder wahr ($w$) oder falsch ($f$) ist; Aussagenvariable $A$; Wahrheitswert $w(A)$
		\item \emph{Syntax:} Induktive Definition korrekt gebildeter aussagenlogischer Formeln F über Variablenmenge $V=\{A, B, \ldots\}$:
		\begin{itemize}
			\item Die Booleschen Wahrheitswerte $w$ und $f$ sind Formeln
			\item Jede Variable aus $V$ ist eine Formel: \emph{Atome}
			\item Negation (NICHT): $\neg F$ ist eine Formel
			\item Konjugation (UND): $(F_1 \land F_2)$ ist eine Formel
			\item Disjunktion (ODER): $(F_1 \lor F_2)$ ist eine Formel
			\item Implikation ("`wenn \ldots, dann"'): $(F_1 \rightarrow F_2)$ ist eine Formel
			\item Äquivalenz ("`genau dann, wenn"'): $(F_1 \leftrightarrow F_2)$ ist eine Formel
			\item andere Verknüpfungen bilden keine Formel
		\end{itemize}
	\end{itemize}
\end{frame}

\begin{frame}{Präzedenzregeln}
	Vereinbarung zur Reduzierung von Klammern
	\begin{itemize}
		\item Bindung (analog "`Punkt vor Strich"'-Rechnung)
		\begin{itemize}
			\item $\neg$ bindet stärker als $\land$
			\item $\land$ bindet stärker als $\lor$
			\item $\lor$ bindet stärker als $\rightarrow$ und $\leftrightarrow$
		\end{itemize}
		\item Operatoren gleicher Stärke: Auswertung linksassoziativ; z.B. \\ $(A \lor B \lor C)$ steht für $((A \lor B) \lor C)$
		\item äußere Klammer weglassen: $(((A \lor B) \rightarrow C) \land B) \mapsto (A \lor B \rightarrow C) \land B$
	\end{itemize}
\end{frame}

\begin{frame}{Semantik}
	aussagenlogischen Formeln wird eine Bedeutung zugeordnet
	\begin{itemize}
		\item Def. Belegung (Interpretation) $I: V \rightarrow \{f, w\}$\\ den Atomen wird jeweils ein konkreter Wahrheitswert zugeordnet
		\item schrittweise (über die Bewertung von Teilformeln) lassen sich dann zusammengesetzte Formeln bewertn:
		\begin{itemize}
			\item $I(\neg F)=w$ falls $I(F)=f$, sonst $I(\neg F)=f$
			\item $I(F \land G)=w$ falls $I(F)=w$ und $I(G)=w$, sonst $I(F \land G)=f$
			\item $I(F \lor G)=w$ falls $I(F)=w$ oder $I(G)=w$, sonst $I(F \lor G)=f$
			\item $I(F \rightarrow G)=w$ falls $I(F)=f$ oder $I(G)=w$, sonst $I(F \rightarrow G)=f$
			\item $I(F \leftrightarrow G)=w$ falls $I(F)=I(G)$, sonst $I(F \leftrightarrow G)=f$
		\end{itemize}
	\end{itemize}
\end{frame}

\begin{frame}{Semantik: Darstellung per Wahrheitstafel}
	Wahrheitstafel enthält zeilenweise alle möglichen Belegungen
	\begin{table}
		\centering
			\begin{tabular}{|c|c|c|c|c|c|c|}
				\hline
				$A$ & $B$ & $\neg A$ & $A \land B$ & $A \lor B$ & $A \rightarrow B$ & $A \leftrightarrow B$ \\
				\hline
				f & f & w & f & f & w & w \\
				\hline
				f & w & w & f & w & w & f \\
				\hline
				w & f & f & f & w & f & f \\
				\hline
				w & w & f & w & w & w & w \\
				\hline
			\end{tabular}
			\caption{Wahrheitstafel der Grundoperationen}
		\end{table}
		logische Äquivalenz $F_1 \equiv F_2$: $F_1$ und $F_2$ haben gleichen Wahrheitswerteverlauf; z.B.\\
		$A \rightarrow B \equiv \neg A \lor B$\\
		$A \leftrightarrow B \equiv (A \rightarrow B) \land (B \rightarrow A)$
\end{frame}

\begin{frame}{Erfüllbarkeit, Tautologie, Kontradiktion}
	\begin{itemize}
		\item Sei $F$ eine Formel und $I$ eine Belegung. Falls $I$ für alle in $F$ vorkommenden Atome definiert ist, so heißt $I$ \underline{zu $F$ passend}
		\item Falls $I$ zu $F$ passend ist, und es gilt $I(F)=w$, dann heißt $I$ \underline{Modell für $F$}; Schreibweise: $I \models F$
		\item $F$ heißt \underline{erfüllbar}, falls $F$ mindestens ein Modell besitzt; sonst heißt $F$ \underline{unerfüllbar} (Kontradiktion; Schreibweise $F\bot$)
		\item $F$ heißt gültig bzw. \underline{Tautologie}, falls jede zu $F$ passende Belegung $I$ ein Modell für $F$ ist; Schreibweise $\models F$ oder $F\top$ \\
		Beispiel: $(A \land B) \rightarrow A$
		\item es gilt: $F$ ist Tautologie gdw. $\neg F$ ist Kontradiktion
	\end{itemize}
\end{frame}

\begin{frame}{Äquivalenz aussagenlogischer Formeln}
	\begin{itemize}
		\item Zwei Formeln $F$ und $G$ heißen äquivalent ($F \equiv G$), falls für alle Belegungen $I$ gilt: $I(F)=I(G)$
		\item Äquivalenz kann bewiesen (oder widerlegt) werden durch aufstellen der jeweiligen Wahrheitstafeln
		\item Beispiele
		\begin{itemize}
			\item $A \rightarrow B \equiv \neg A \lor B$
			\item $A \leftrightarrow B \equiv (A \land B) \lor (\neg A \land \neg B) \equiv (A \rightarrow B) \land (B \rightarrow A)$
		\end{itemize}
		\item Ersetzbarkeitstheorem:\\
		enthält eine Formel $F$ eine Teilformel $G$ und wird $G$ durch eine äquivalente Formel ersetzt, so entsteht eine zu $F$ äquivalente Formel
	\end{itemize}
\end{frame}

\begin{frame}{Äquivalenzen als Rechenregeln}
	\begin{itemize}
		\item Idempotenz
		\begin{itemize}
			\item $(A \land A) \equiv A$
			\item $(A \lor A) \equiv A$
		\end{itemize}
		\item Kommutativgesetz
		\begin{itemize}
			\item $(A \land B) \equiv (B \land A)$
			\item $(A \lor B) \equiv (B \lor A)$
		\end{itemize}
		\item Assoziativgesetz
		\begin{itemize}
			\item $(A \land (B \land C)) \equiv ((A \land B) \land C)$
			\item $(A \lor (B \lor C)) \equiv ((A \lor B) \lor C)$
		\end{itemize}
		\item Distributivgesetz
		\begin{itemize}
			\item $(A \land (B \lor C)) \equiv ((A \land B) \lor (A \land C))$
			\item $(A \lor (B \land C)) \equiv ((A \lor B) \land (A \lor C))$
		\end{itemize}
	\end{itemize}
\end{frame}

\begin{frame}{Äquivalenzen als Rechenregeln (Fortsetzung)}
	\begin{itemize}
		\item Absorptionsgesetz
		\begin{itemize}
			\item $(A \land (A \lor B)) \equiv A$
			\item $(A \lor (A \land B)) \equiv A$
		\end{itemize}
		\item Doppelnegation
		\begin{itemize}
			\item $\neg \neg A \equiv A$
		\end{itemize}
		\item deMorgansche Regeln
		\begin{itemize}
			\item $\neg (A \land B) \equiv (\neg A \lor \neg B)$
			\item $\neg (A \lor B) \equiv (\neg A \land \neg B)$
		\end{itemize}
		\item Tautologieregeln: sei $A$ Tautologie; $A\top$ bzw. $\models A$
		\begin{itemize}
			\item $(A \lor B) \equiv A$; $(A \land B) \equiv B$
		\end{itemize}
		\item Kontradiktionsregeln: sei $A$ Kontradiktion; $A\bot$
		\begin{itemize}
			\item $(A \lor B) \equiv B$; $(A \land B) \equiv A$
		\end{itemize}
	\end{itemize}
\end{frame}

\begin{frame}{Beispiel: direkter Beweis der Gültigkeit einer Formel (als Alternative zur Wahrheitstafel)}
	\begin{columns}
		\column{.5\textwidth}
		\begin{eqnarray*}
				F &=& (A \land (A \rightarrow B)) \rightarrow B\\
				  &\equiv& \neg (A \land (A \rightarrow B)) \lor B\\
				  &\equiv& \neg (A \land (\neg A \lor B)) \lor B\\
				  &\equiv& \neg A \lor \neg(\neg A \lor B) \lor B\\
				  &\equiv& \neg A \lor (A \land \neg B) \lor B\\
				  &\equiv& \neg A \lor (A \lor B) \land (\neg B \lor B)\\
				  &\equiv& \neg A \lor (A \lor B) \land w\\
				  &\equiv& \neg A \lor A \lor B\\
				  &\equiv& w \lor B\\
				  &\equiv& w
		\end{eqnarray*}
		\column{.5\textwidth}
		\begin{table}
			\begin{tabular}{|c|c|c|c|c|}
				\hline
				$A$ & $B$ & $A \rightarrow B$ & $A \land (A \rightarrow B)$ & $F$ \\
				\hline
				f & f & w & f & w \\
				\hline
				f & w & w & f & w \\
				\hline
				w & f & f & f & w \\
				\hline
				w & w & w & w & w \\
				\hline
			\end{tabular}
			\caption{Wahrheitstafel}
			\label{tblWahrheitswerteGrundoperationen}
		\end{table}
	\end{columns}		
\end{frame}

\begin{frame}{Normalformen}
	\begin{itemize}
		\item \emph{Literal:} Aussagenvariable $A$ ("`positives Literal"') oder negierte Aussagenvariable $\neg A$ ("`negatives Literal"')
		\item \emph{Negationsnormalform (NNF)}: Formel $F$ ist in NNF, falls $\rightarrow$ und $\leftrightarrow$ aufgelöst sind und alle Negationszeichen $\neg$ unmittelbar vor einer Aussagenvariable stehen.\\
		beachte: NNF ist nicht eindeutig
		\item \emph{Monom:} Konjunktion von Literalen, d.h. Formel der Art\\
		$\bigwedge_iL_i=L_1\land L_2\land\ldots\land L_k$
		\item \emph{Klausel:} Disjunktion von Literalen, d.h. Formel der Art\\
		$\bigvee_iL_i=L_1\lor L_2\lor\ldots\lor L_k$
	\end{itemize}
\end{frame}

\begin{frame}{Disjunktive und konjunktive Normalform}
	\begin{itemize}
		\item Disjunktive Normalform \emph{DNF:} Disjunktion von Monomen\\
		$F=\bigvee_i\left(\bigwedge_jL_{ij}\right)$
		\item Konjunktive Normalform \emph{KNF:} Konjunktion von Klauseln\\
		$F=\bigwedge_i\left(\bigvee_jL_{ij}\right)$
		\item Für jede Formel existiert eine äquivalente Formel in KNF und eine äquivalente Formel in DNF
		\begin{itemize}
			\item Erzeuge NNF $\mapsto$ DNF bzw. KNF\\
			DNF: ersetze Ausdrücke der Art $(F \land (G \lor H))$ durch $(F \land G) \lor (F \land H)$ solange solche Teilformen existieren\\
			KNF: ersetze Ausdrücke der Art $(F \lor (G \land H))$ durch $(F \lor G) \land (F \lor H)$ solange solche Teilformeln existieren
		\end{itemize}
	\end{itemize}
\end{frame}

\begin{frame}{Konstruktion kanonische DNF/KNF aus Wahrheitstafel}
	\begin{itemize}
		\item \emph{Kanonische} DNF bzw. KNF: jede Aussagenvariable kommt in jedem Monom bzw. jeder Klausel genau einmal vor (positiv oder negativ)
		\item Kanonische NF lassen sich aus der Wahrheitstafel der Formel ablesen
		\begin{itemize}
			\item Kanonische DNF (KDNF)
			\begin{itemize}
				\item Je ein Monom für jede Belegung mit Wahrheitswert $w$
				\item Setze $L_j=A_j$, falls $w(A_j)=w$, $L_j=\neg A_j$ sonst
			\end{itemize}
			\item Kanonische KNF (KKNF)
			\begin{itemize}
				\item Je eine Klausel für jede Belegung mit Wahrheitswert $f$
				\item Setze $L_j=\neg A_j$, falls $w(A_j)=w$, $L_j=A_j$ sonst
			\end{itemize}
		\end{itemize}
		\item Übungsbeispiel (KNF, aber nicht kanonisch)\\
		$F=\neg B \land (A \lor \neg C)$
	\end{itemize}
\end{frame}