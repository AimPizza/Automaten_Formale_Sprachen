\section{Kellerautomaten}

\begin{frame}{Kellerautomaten (PDA, i.A. nichtdeterministisch)}
	\begin{itemize}
		\item Grundidee Kellerautomat (PDA, push-down automaton)
		\begin{itemize}
			\item FA wird erweitert um einen zusätzlichen Kellerspeicher (Stack, Stapel)
			mit unbeschränkter Kapazität
			\item Stack: Datenstruktur mit den Methoden pop und push (LIFO-Prinzip)
			\item pop: oberstes Zeichen wird gelesen und gelöscht
			\item push: neues Zeichen wird auf den Stack geschrieben
		\end{itemize}
		\item Konfigurationsübergänge
		\begin{itemize}
			\item Nächstes Zeichen des Eingabewortes sowie oberstes Symbol auf Stack
			werden gelesen
			\item abhängig davon erfolgt Zustandsübergang, und es werden (i.A.
			mehrere) push-Operationen ausgeführt
			\item möglich sind auch $\varepsilon$-Übergänge: Zustandsübergang und Stack-
			Operationen ohne Weiterlesen des Eingabewortes
		\end{itemize}
	\end{itemize}
\end{frame}

\begin{frame}{Spezifikation PDA}
	\begin{itemize}
		\item Spezifikation nichtdeterministischer PDA\\
		$A_{PDA}=(S, \Sigma, \Gamma, \delta, s_0, \#, F)$
		\begin{itemize}
			\item $S$: Zustandsmenge
			\item $\Sigma$: Terminalalphabet
			\item $\Gamma$: Kelleralphabet
			\item $\delta$: Zustandsüberführungsrelation\\
			$\delta \subseteq S \times \Sigma \times \Gamma \times \Sigma \times \Gamma^*$
			\item $\#$: Kellerboden-Symbol, $\# \in \Gamma$
			\item $s_0$: Startzustand, $s_0 \in S$
			\item $F$: Menge von Endzuständen
		\end{itemize}
		\item Konfiguration: $(s, u, \gamma)$
		\item Konfigurationsübergang: $(s_1, av, A\beta) \mapsto (s_2, v, \alpha\beta)$
	\end{itemize}
\end{frame}