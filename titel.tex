%title page
\begin{frame}
	\titlepage
\end{frame}

%\begin{frame}
%	\frametitle{Danksagung}
%	Diese Vorlesung basiert maßgeblich auf dem Script von\\
%	Prof. Dr. habil. Jochen Kripfganz\\
%	Dozent für Theoretische Informatik an der BA Leipzig bis 2021
%\end{frame}

\begin{frame}{Organisatorisches}
	\begin{itemize}
		\item Vorlesungen
		\item Übungen
		\item Fragen zum Lehrstoff
		\item besondere Bedürfnisse
	\end{itemize}
\end{frame}

\AtBeginSection{
	\begin{frame}
		\sectionpage
		\tableofcontents[sectionstyle=hide/hide,subsectionstyle=show/show/hide]
	\end{frame}
}

\begin{frame}
	\frametitle{Literatur}
	\begin{columns}
		\column{.8\textwidth}
			\begin{itemize}
				\item Script zur Vorlesung \\
				{\small \url{https://github.com/bluepoke/Automaten_Formale_Sprachen/releases/latest}}
				\item Grundlage der Vorlesung
				\begin{itemize}
					\item Schöning, U.: \textit{Logik für Informatiker}
					\item Schöning, U.: \textit{Theoretische Informatik - kurz gefasst}
				\end{itemize}
				\item Weitere Literatur
				\begin{itemize}
					\item Vossen, G.; Witt, K. U.: \textit{Grundlagen der Theoretischen Informatik mit Anwendungen}
					\item Hollas, B.: \textit{Grundkurs Theoretische Informatik}
					\item Erk, K.;  Priese, L.: \textit{Theoretische Informatik}
				\end{itemize}
			\end{itemize}
		\column{.2\textwidth}
				\qrcode[height=0.9\textwidth]{https://github.com/bluepoke/Automaten_Formale_Sprachen/releases/latest}
	\end{columns}
\end{frame}

%table of contents
\begin{frame}
	\frametitle{Gliederung}
	\tableofcontents
\end{frame}