\documentclass[a4paper,12pt]{article}
\usepackage{fancyhdr}
\usepackage{fancyheadings}
\usepackage[ngerman,german]{babel}
\usepackage{german}
\usepackage[utf8]{inputenc}
%\usepackage[latin1]{inputenc}
\usepackage[active]{srcltx}
\usepackage{algorithm}
\usepackage[noend]{algorithmic}
\usepackage{amsmath}
\usepackage{amssymb}
\usepackage{amsthm}
\usepackage{bbm}
\usepackage{enumerate}
\usepackage{graphicx}
\usepackage{ifthen}
\usepackage{listings}
\usepackage{struktex}
\usepackage{hyperref}

% table placement
\usepackage{placeins}

\newcommand{\Fach}{Theoretische Informatik I - Automaten und Formale Sprachen}
\newcommand{\Name}{}
\newcommand{\Seminargruppe}{5CS21-2} %  <-- UPDATE ME
\newcommand{\Uebung}{4} %  <-- UPDATE ME
\newcommand{\Uebungstitel}{Eigenschaften regulärer bzw. kontextfreier Sprachen, Chomsky-Normalform und CYK-Algorithmus} %  <-- UPDATE ME
\newcommand{\Besprechungstermin}{03.03.2022} %  <-- UPDATE ME

\setlength{\parindent}{0em}
\topmargin -1.0cm
\oddsidemargin 0cm
\evensidemargin 0cm
\setlength{\textheight}{9.2in}
\setlength{\textwidth}{6.0in}

%%%%%%%%%%%%%%%
%% Aufgaben-COMMAND
\newcommand{\Aufgabe}[1]{
	{
		\vspace*{0.5cm}
		\textsf{\textbf{Aufgabe #1}}
		\vspace*{0.2cm}
		
	}
}
%%%%%%%%%%%%%%
\hypersetup{
	pdftitle={\Fach{}: Übungsblatt \Uebung{}},
	pdfauthor={\Name},
	pdfborder={0 0 0}
}

\lstset{ %
	language=java,
	basicstyle=\footnotesize\tt,
	showtabs=false,
	tabsize=2,
	captionpos=b,
	breaklines=true,
	extendedchars=true,
	showstringspaces=false,
	flexiblecolumns=true,
}

\title{Übungsblatt \Uebung{}}
\author{\Name{}}

\begin{document}
	\thispagestyle{fancy}
	\chead{\sf \large \Fach{} \\ \small \Name{}}
	\chead{\sf \large \Fach{} \\ \small Seminargruppe \Seminargruppe{}}
	\vspace*{0.2cm}
	\begin{center}
		\LARGE \sf \textbf{Übung \Uebung{}} \\
		\vspace*{0.4cm}
		\Large \sf \textbf{\Uebungstitel}\\
		\vspace*{0.4cm}
		\normalsize \rm Besprechungstermin: \Besprechungstermin
	\end{center}
	\vspace*{0.2cm}
	
	%%%%%%%%%%%%%%%%%%%%%%%%%%%%%%%%%%%%
	%% Aufgaben %%%%%%%%%%%%%%%%%%%%%%%%
	%%%%%%%%%%%%%%%%%%%%%%%%%%%%%%%%%%%%
	
	Hinweis: Zur Lösung der Aufgaben 1 bis 5 dieser Serie greifen Sie auf das Pumping-Lemma bzw. die Abschlusseigenschaften der jeweiligen Sprachklasse zurück.
	
	\Aufgabe{1}
	Zeigen Sie, dass die Sprache $L= \{scs \mid s \in \{a, b\}^* \}$ nicht regulär ist.
	
	\Aufgabe{2}
	Zeigen Sie, dass die Sprache $L= \{scs \mid s \in \{a, b\}^* \}$ nicht kontextfrei ist.
	
	\Aufgabe{3}
	$L$ sei regulär und werde durch einen DFA mit 4 Zuständen akzeptiert. $L$ enthalte ein Wort der Länge 4. Ist $L$ unendlich? Begründen Sie die Antwort.
	
	\Aufgabe{4}
	Seien $L_1, L_2 \subseteq \Sigma^*$ zwei Sprachen. $L_1$ und $L=L_1 \cup L_2$ seien regulär. Es gelte $L_1 \cap L_2 = \varnothing$. Beweisen Sie: $L_2$ ist regulär.
	
	\Aufgabe{5}
	Zeigen Sie: die folgende Sprache ist nicht kontextfrei: $$L=\{w \in \{a, b, c\}^* \mid |w|_a=|w|_b=|w|_c\}$$ Dabei bezeichnet $|w|_a$ die Anzahl von $a$'s im Wort $w$.
	
	\pagebreak
	\Aufgabe{6}
	Gegeben sei
	$$G=(\{S,A,B\},\{a,b,c\},P,S)$$
	mit der Produktionsmenge $P$:
	\begin{align*}
		S \rightarrow & aAb \mid aBb\\
		A \rightarrow & S \mid aaSc \mid B \mid \varepsilon\\
	\end{align*}
	Überführen Sie diese Grammatik in Chomsky-Normalform.
	
	\Aufgabe{7}
	Gegeben sei folgende Grammatik $G$ in Chomsky-Normalform:
	\begin{align*}
		S \rightarrow & AB\\
		A \rightarrow & X_aX_b | X_aC\\
		C \rightarrow & AX_a\\
		B \rightarrow & c | X_cB\\
		X_a \rightarrow & a\\
		X_b \rightarrow & b\\
		X_c \rightarrow & c\\
	\end{align*}
	Prüfen Sie mittels CYK-Algorithmus, ob das Wort $aabacc$ in $L(G)$ enthalten ist.
\end{document}

