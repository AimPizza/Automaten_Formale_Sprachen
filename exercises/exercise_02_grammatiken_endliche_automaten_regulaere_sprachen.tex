\documentclass[a4paper,12pt]{article}
\usepackage{fancyhdr}
\usepackage{fancyheadings}
\usepackage[ngerman,german]{babel}
\usepackage{german}
\usepackage[utf8]{inputenc}
%\usepackage[latin1]{inputenc}
\usepackage[active]{srcltx}
\usepackage{algorithm}
\usepackage[noend]{algorithmic}
\usepackage{amsmath}
\usepackage{amssymb}
\usepackage{amsthm}
\usepackage{bbm}
\usepackage{enumerate}
\usepackage{graphicx}
\usepackage{ifthen}
\usepackage{listings}
\usepackage{struktex}
\usepackage{hyperref}

% table placement
\usepackage{placeins}

\newcommand{\Fach}{Theoretische Informatik I - Automaten und Formale Sprachen}
\newcommand{\Name}{}
\newcommand{\Seminargruppe}{5CS21-2}
\newcommand{\Uebung}{2} %  <-- UPDATE ME
\newcommand{\Uebungstitel}{Grammatiken, endliche Automaten, reguläre Sprachen} %  <-- UPDATE ME
\newcommand{\Besprechungstermin}{08.02.2022}

\setlength{\parindent}{0em}
\topmargin -1.0cm
\oddsidemargin 0cm
\evensidemargin 0cm
\setlength{\textheight}{9.2in}
\setlength{\textwidth}{6.0in}

%%%%%%%%%%%%%%%
%% Aufgaben-COMMAND
\newcommand{\Aufgabe}[1]{
	{
		\vspace*{0.5cm}
		\textsf{\textbf{Aufgabe #1}}
		\vspace*{0.2cm}
		
	}
}
%%%%%%%%%%%%%%
\hypersetup{
	pdftitle={\Fach{}: Übungsblatt \Uebung{}},
	pdfauthor={\Name},
	pdfborder={0 0 0}
}

\lstset{ %
	language=java,
	basicstyle=\footnotesize\tt,
	showtabs=false,
	tabsize=2,
	captionpos=b,
	breaklines=true,
	extendedchars=true,
	showstringspaces=false,
	flexiblecolumns=true,
}

\title{Übungsblatt \Uebung{}}
\author{\Name{}}

\begin{document}
	\thispagestyle{fancy}
	\chead{\sf \large \Fach{} \\ \small \Name{}}
	\chead{\sf \large \Fach{} \\ \small Seminargruppe \Seminargruppe{}}
	\vspace*{0.2cm}
	\begin{center}
		\LARGE \sf \textbf{Übung \Uebung{}} \\
		\vspace*{0.4cm}
		\Large \sf \textbf{\Uebungstitel}\\
		\vspace*{0.4cm}
		\normalsize \rm Besprechungstermin: \Besprechungstermin
	\end{center}
	\vspace*{0.2cm}
	
	%%%%%%%%%%%%%%%%%%%%%%%%%%%%%%%%%%%%
	%% Aufgaben %%%%%%%%%%%%%%%%%%%%%%%%
	%%%%%%%%%%%%%%%%%%%%%%%%%%%%%%%%%%%%
	
	\Aufgabe{1}
	Welche Sprache $L(G)$ wird durch die folgende Grammatik $G$ erzeugt? Ist die Grammatik eindeutig? Falls nein: lässt sich für diese Sprache eine eindeutige Grammatik angeben? Von welchem Typ ist die Sprache?
	
	\begin{align*}
		G&=\left(\left\{S,A,B\right\},\left\{a,b,c\right\},P,S\right)\\
		P&=\left\{S \rightarrow aB \mid Ac, A \rightarrow ab, B \rightarrow bc\right\}
	\end{align*}

	\Aufgabe{2}
	Ist die Sprache $L = \left\{0^{2n} \mid n \in \mathbb{N}_0\right\}$ regulär? Beantworten Sie die Frage vom Standpunkt einer erzeugenden Grammatik, sowie von dem eines akzeptierenden Automaten.
	
	\Aufgabe{3}
	Konstruieren Sie einen DFA, der alle Worte über $\{0,1\}$ erkennt, die den String $101$ \underline{nicht} als Teilwort enthalten.
	
	\Aufgabe{4}
	Konstruieren Sie einen DFA der die Menge aller natürlichen Zahlen kongruent zu Null (mod 5) akzeptiert. Eingabe ist jeweils eine natürliche Zahl in Dualdarstellung.
	
	\Aufgabe{5}
	Geben Sie eine Typ-3 Grammatik an, die die durch folgenden Automaten akzeptierte Sprache produziert. Beschreiben Sie die akzeptierte Sprache verbal. Ermitteln sie dazu gegebenenfalls den Minimalautomaten zu $A$.
	
	\begin{align*}
		A&=\left(\left\{s_0,s_1,s_2,s_3,s_4\right\},\left\{0,1\right\},\delta,s_0,\left\{s_3,s_4\right\}\right)\\
		\delta &=\{(s_0,0,s_1),(s_0,1,s_2),(s_1,0,s_2),(s_1,1,s_3),(s_2,0,s_1),(s_2,1,s_3),\\
		&\qquad(s_3,0,s_1),(s_3,1,s_4),(s_4,0,s_2),(s_4,1,s_4)\}
	\end{align*}

	\pagebreak
	
	\Aufgabe{6}
	Konstruieren Sie zu folgendem nichtdeterministischem endlichem Automaten
	$$A=(\{p,q,r\}, \{a,b\}, \delta, p, \{r\})$$
	mit der mengenwertigen Übergangsfunktion $\delta$
	\begin{center}
		\begin{tabular}{|l|l|l|l|}
			\hline
			$\delta$ & $p$ & $q$ & $r$ \\
			\hline
			$a$ & $\{p,q\}$ & $\{p,q,r\}$ & $\{\}$ \\
			\hline
			$b$ & $\{r\}$ & $\{\}$ & $\{q,r\}$ \\
			\hline
		\end{tabular}
	\end{center}
	einen äquivalenten deterministischen Automaten.\\
	Ermitteln Sie zu diesem DFA den Minimalautomaten. 
\end{document}

