\documentclass[a4paper,12pt]{article}
\usepackage{fancyhdr}
\usepackage{fancyheadings}
\usepackage[ngerman,german]{babel}
\usepackage{german}
\usepackage[utf8]{inputenc}
%\usepackage[latin1]{inputenc}
\usepackage[active]{srcltx}
\usepackage{algorithm}
\usepackage[noend]{algorithmic}
\usepackage{amsmath}
\usepackage{amssymb}
\usepackage{amsthm}
\usepackage{bbm}
\usepackage{enumerate}
\usepackage{graphicx}
\usepackage{ifthen}
\usepackage{listings}
\usepackage{struktex}
\usepackage{hyperref}

% table placement
\usepackage{placeins}

\newcommand{\Fach}{Theoretische Informatik I - Automaten und Formale Sprachen}
\newcommand{\Name}{}
\newcommand{\Seminargruppe}{5CS21-1} %  <-- UPDATE ME
\newcommand{\Uebung}{5} %  <-- UPDATE ME
\newcommand{\Uebungstitel}{Turingmaschinen} %  <-- UPDATE ME
\newcommand{\Besprechungstermin}{1. Dezember 2021} %  <-- UPDATE ME

\setlength{\parindent}{0em}
\topmargin -1.0cm
\oddsidemargin 0cm
\evensidemargin 0cm
\setlength{\textheight}{9.2in}
\setlength{\textwidth}{6.0in}

%%%%%%%%%%%%%%%
%% Aufgaben-COMMAND
\newcommand{\Aufgabe}[1]{
	{
		\vspace*{0.5cm}
		\textsf{\textbf{Aufgabe #1}}
		\vspace*{0.2cm}
		
	}
}
%%%%%%%%%%%%%%
\hypersetup{
	pdftitle={\Fach{}: Übungsblatt \Uebung{}},
	pdfauthor={\Name},
	pdfborder={0 0 0}
}

\lstset{ %
	language=java,
	basicstyle=\footnotesize\tt,
	showtabs=false,
	tabsize=2,
	captionpos=b,
	breaklines=true,
	extendedchars=true,
	showstringspaces=false,
	flexiblecolumns=true,
}

\title{Übungsblatt \Uebung{}}
\author{\Name{}}

\begin{document}
	\thispagestyle{fancy}
	\chead{\sf \large \Fach{} \\ \small \Name{}}
	\chead{\sf \large \Fach{} \\ \small Seminargruppe \Seminargruppe{}}
	\vspace*{0.2cm}
	\begin{center}
		\LARGE \sf \textbf{Übung \Uebung{}} \\
		\vspace*{0.4cm}
		\Large \sf \textbf{\Uebungstitel}\\
		\vspace*{0.4cm}
		\normalsize \rm Besprechungstermin: \Besprechungstermin
	\end{center}
	\vspace*{0.2cm}
	
	%%%%%%%%%%%%%%%%%%%%%%%%%%%%%%%%%%%%
	%% Aufgaben %%%%%%%%%%%%%%%%%%%%%%%%
	%%%%%%%%%%%%%%%%%%%%%%%%%%%%%%%%%%%%
	
	\Aufgabe{1}
	
	Konstruieren Sie eine Turingmaschine, die das Wortproblem für die Sprache
	$$L=\{a^nb^nc^nd^n \mid n\geq 0 \}$$ entscheidet.
	
	\Aufgabe{2}
	
	Konstruieren Sie eine Turingmaschine, die das Wortproblem für die Sprache
	$$L=\{w\tilde{w} \mid w \in \{a,b\}^*\}$$ entscheidet.
	
	\Aufgabe{3}
	
	Konstruieren Sie eine 2-Band-Turingmaschine, die das Wortproblem für die Sprache
	$$L=\{ww \mid w \in \{a,b\}^+\}$$ entscheidet.
	
	\Aufgabe{4}
	
	Konstruieren Sie eine Turingmaschine, die das Wortproblem für die Sprache
	$$L=\{w \in \{a,b,c\}^*\mid |w|_a=|w|_b=|w|_c\}$$ entscheidet. Hierbei ist $|w|_a$ die Anzahl $a$'s im Wort $w$.
\end{document}

